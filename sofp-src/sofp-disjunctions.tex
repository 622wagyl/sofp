
\chapter{The logic of types. I. Disjunctive types}

Disjunctive types describe values that belong to a disjoint set of
alternatives. 

To see how Scala implements disjunctive types, we need to begin by
looking at ``case classes''.

\section{Scala's case classes}

\subsection{Tuple types with names}

It is often helpful to use names for the different parts of a tuple.
Suppose that some program represents the size and the color of socks
with the tuple type \lstinline!(Double, String)!. What if the same
tuple type \lstinline!(Double, String)! is used in another place
in the program to mean the amount paid and the payee? A programmer
could mix the two values by mistake, and it would be hard to find
out why the program incorrectly computes, say, the total amount paid.
\begin{lstlisting}
def totalAmountPaid(ps: Seq[(Double, String)]): Double = ps.map(_._1).sum
val x = (10.5, "white")       // Sock size and color.
val y = (25.0, "restaurant")  // Payment amount and payee.

scala> totalAmountPaid(Seq(x, y)) // Nonsense.
res0: Double = 35.5
\end{lstlisting}

We would prevent this kind of mistake if we could use two \emph{different}
types, with names such as \lstinline!MySock! and \lstinline!Payment!,
for the two kinds of data. There are  three basic ways of defining
a new named type in Scala: using a type alias, using a class (or ``trait''),
and using an \index{opaque type}opaque type. 

Opaque types (hiding a type under a new name) is a feature of a future
version of Scala 3; so we focus on type aliases and case classes.

A \textbf{type alias}\index{type alias} is an alternative name for
an existing (already defined) type. We could use type aliases in our
example to add clarity to the code:
\begin{lstlisting}
type MySockTuple = (Double, String)
type PaymentTuple = (Double, String)

scala> val s: MySockTuple = (10.5, "white")
s: MySockTuple = (10.5,white)

scala> val p: PaymentTuple = (25.0, "restaurant")
p: PaymentTuple = (25.0,restaurant)
\end{lstlisting}
But type aliases do not prevent mix-up errors:
\begin{lstlisting}
scala> totalAmountPaid(Seq(s, p)) // Nonsense again.
res1: Double = 35.5
\end{lstlisting}

Scala's \textbf{case classes}\index{case class} can be seen as ``tuples
with names''. A case class is equivalent to a tuple type that has
a name chosen when we define the case class. Also, each part of the
case class will have a separate name that we must choose. This is
how to define case classes for the example with socks and payments:
\begin{lstlisting}
case class MySock(size: Double, color: String)
case class Payment(amount: Double, name: String)

scala> val sock = MySock(10.5, "white")
sock: MySock = MySock(10.5,white)

scala> val paid = Payment(25.0, "restaurant")
paid: Payment = Payment(25.0,restaurant)                                  ^ 
\end{lstlisting}
This code defines new types named \lstinline!MySock! and \lstinline!Payment!.
Values of type \lstinline!MySock! are written as \lstinline!MySock(10.5, "white")!,
which is similar to writing the tuple \lstinline!(10.5, "white")!
except for adding the name \lstinline!MySock! in front of the tuple.

To access the parts of a case class, we use the part names:
\begin{lstlisting}
scala> sock.size
res2: Double = 10.5

scala> paid.amount
res3: Double = 25.0
\end{lstlisting}
The mix-up error is now a \index{type error}type error detected by
the compiler:
\begin{lstlisting}
def totalAmountPaid(ps: Seq[Payment]): Double = ps.map(_.amount).sum

scala> totalAmountPaid(Seq(paid, paid))
res4: Double = 50.0

scala> totalAmountPaid(Seq(sock, paid))
<console>:19: error: type mismatch;
 found   : MySock
 required: Payment
       totalAmountPaid(Seq(sock, paid))
                           ^
\end{lstlisting}
A function whose argument is of type \lstinline!MySock! cannot be
applied to an argument of type \lstinline!Payment!. Case classes
with different names are \emph{different types}, even if they contain
the same parts. 

Just as tuples can have any number of parts, case classes can have
any number of parts, but the part names must be distinct, for example:
\begin{lstlisting}
case class Person(firstName: String, lastName: String, age: Int)

scala> val noether = Person("Emmy", "Noether", 137)
einstein: Person = Person(Emmy,Noether,137)

scala> noether.firstName
res5: String = Emmy

scala> noether.age
res6: Int = 137
\end{lstlisting}
This data type carries the same information as a tuple \lstinline!(String, String, Int)!.
However, the declaration of a \lstinline!case class Person! gives
the programmer several features that make working with the tuple's
data more convenient and less error-prone.

Some (or all) part names may be specified when creating a case class
value:
\begin{lstlisting}[extendedchars=true,mathescape=true]
scala> val poincar$\text{\'e}$ = Person(firstName = "Henri", lastName = "Poincar$\text{\color{mauve}\'e}$", 165)
poincar$\text{\'{e}}$: Person = Person(Henri,Poincar$\text{\'e}$,165)
\end{lstlisting}
It is a type error to use wrong types with a case class:
\begin{lstlisting}
scala> val p = Person(140, "Einstein", "Albert")
<console>:13: error: type mismatch;
 found   : Int(140)
 required: String
       val p = Person(140, "Einstein", "Albert")
                      ^
<console>:13: error: type mismatch;
 found   : String("Albert")
 required: Int
       val p = Person(140, "Einstein", "Albert")
                                       ^
\end{lstlisting}
This error is due to an incorrect order of parts when creating a case
class value. However, parts can be specified in any order when using
part names:
\begin{lstlisting}
scala> val p = Person(age = 137, lastName = "Noether", firstName = "Emmy")
p: Person = Person(Emmy,Noether,137)
\end{lstlisting}
A part of a case class can have the type of another case class, creating
a type similar to a nested tuple:
\begin{lstlisting}
case class BagOfSocks(sock: MySock, count: Int)
val bag = BagOfSocks(MySock(10.5, "white"), 6)

scala> bag.sock.size
res7: Double = 10.5
\end{lstlisting}


\subsection{Case classes with type parameters}

Type classes can be defined with \index{type parameter}type parameters.
As an example, consider a generalization of \lstinline!MySock! where,
in addition to the size and color, an ``extended sock'' holds another
value. We could define several specialized case classes,
\begin{lstlisting}
case class MySock_Int(size: Double, color: String, value: Int)
case class MySock_Boolean(size: Double, color: String, value: Boolean)
\end{lstlisting}
but it is better to define a single parameterized case class
\begin{lstlisting}
case class MySockX[A](size: Double, color: String, value: A)
\end{lstlisting}
This case class can accommodate every type \lstinline!A!. We may
now create values of \lstinline!MySockX! containing a \lstinline!value!
of any given type,
\begin{lstlisting}
scala> val s = MySockX(10.5, "white", 123)
s: MySockX[Int] = MySockX(10.5,white,123) 
\end{lstlisting}
Because \lstinline!123! has type \lstinline!Int!, the type parameter
\lstinline!A! in \lstinline!MySockX[A]! was automatically set to
the type \lstinline!Int!. 

Each time we create a value of type \lstinline!MySockX!, a specific
type will have to be used instead of the type parameter \lstinline!A!.
In other words, we can only create values of types \lstinline!MySockX[Int]!,
\lstinline!MySockX[String]!, etc. If we want to be explicit, we may
write
\begin{lstlisting}
scala> val s = MySockX[String](10.5, "white", "last pair")
s: MySockX[String] = MySockX(10.5,white,last pair) 
\end{lstlisting}
However, we can write code working with \lstinline!MySockX[A]! \textbf{parametrically}\index{parametric code},
that is, keeping the type parameter \lstinline!A! in the code. For
example, a function that checks whether a sock of type \lstinline!MySockX[A]!
fits my foot can be written as
\begin{lstlisting}
def fitsMe[A](sock: MySockX[A]): Boolean = (sock.size >= 10.5 && sock.size <= 11.0)
\end{lstlisting}
This function is defined for all types \lstinline!A! at once, because
its code works in the same way regardless of what \lstinline!A! is.
Scala will set the type parameter automatically when we apply \lstinline!fitsMe!
to an argument:
\begin{lstlisting}
scala> fitsMe(MySockX(10.5, "blue", List(1,2,3))) // Type parameter A = List[Int].
res0: Boolean = true
\end{lstlisting}
This code forces the type parameter \lstinline!A! to be \lstinline!List[Int]!,
and so we may omit the type parameter of \lstinline!fitsMe!. When
types become more complicated, it may be helpful to write out some
type parameters. The compiler can detect a mismatch between the type
parameter \lstinline!A = List[Int]! used in the ``sock'' value
and the type parameter \lstinline!A = Int! in the function \lstinline!fitsMe!:
\begin{lstlisting}
scala> fitsMe[Int](MySockX(10.5, "blue", List(1,2,3)))
<console>:15: error: type mismatch;
 found   : List[Int]
 required: Int
       fitsMe[Int](MySockX(10.5, "blue", List(1,2,3)))
                                             ^ 
\end{lstlisting}

Case classes may have several type parameters, and the types of the
parts may use these type parameters. Here is an artificial example
of a case class using type parameters in different ways,
\begin{lstlisting}
case class Complicated[A,B,C,D](x: (A, A), y: (B, Int) => A, z: C => C)
\end{lstlisting}
This case class contains parts of different types that use the type
parameters \lstinline!A!, \lstinline!B!, \lstinline!C! in tuples
and functions. The type parameter \lstinline!D! is not used at all;
this is allowed (and occasionally useful).

A type with type parameters, such as \lstinline!MySockX! or \lstinline!Complicated!,
is called a \index{type constructor}\textbf{type constructor}. A
type constructor ``constructs'' a new type, such as \lstinline!MySockX[Int]!,
from a given type parameter \lstinline!Int!. Values of type \lstinline!MySockX!
cannot be created without setting the type parameter. So, it is important
to distinguish the type constructor, such as \lstinline!MySockX!,
from a type that can have values, such as \lstinline!MySockX[Int]!.

\subsection{Tuples with one part and with zero parts}

Let us compare tuples and case classes more systematically.

Parts of a case class are accessed by name with a dot syntax, for
example \lstinline!sock.color!. Parts of a tuple are accessed with
the accessors such as \lstinline!x._1!. This syntax is the same as
that for a case class whose parts have names \lstinline!_1!, \lstinline!_2!,
etc. So, it appears that tuple parts \emph{do} have names in Scala,
although those names are always automatically chosen as \lstinline!_1!,
\lstinline!_2!, etc. Tuple types are also automatically named in
Scala as \lstinline!Tuple2!, \lstinline!Tuple3!, etc., and they
are parameterized, since each part of the tuple may be of any chosen
type. A tuple type expression such as \lstinline!(Int, String)! is
just a special syntax for the parameterized type \lstinline!Tuple2[Int, String]!.
One could define the tuple types as case classes like this,
\begin{lstlisting}
case class Tuple2[A, B](_1: A, _2: B)
case class Tuple3[A, B, C](_1: A, _2: B, _3: C)  // And so on with Tuple4, Tuple5...
\end{lstlisting}
if these types were not already defined in the Scala library.

Proceeding systematically, we ask whether tuple types can have just
one part or even no parts. Indeed, Scala defines \lstinline!Tuple1[A]!
as a tuple with a single part. (This type is occasionally used in
practice.) 

The tuple with zero parts also exists and is called \lstinline!Unit!
(rather than ``\lstinline!Tuple0!''). The syntax for the value
of the \lstinline!Unit! type is the empty tuple, \lstinline!()!.
It is clear that there is \emph{only one} value, \lstinline!()!,
of this type; this explains the name ``unit\index{unit type}''. 

At first sight, the \lstinline!Unit! type may appear to be completely
useless: it is a tuple that contains \emph{no} \emph{data}. It turns
out, however, that the \lstinline!Unit! type is important in functional
programming, and it is used as a type \emph{guaranteed} to have only
a single distinct value. This chapter will show some examples of using
the \lstinline!Unit! type.

Case classes may have one part or zero parts, similarly to the one-part
and zero-part tuples:
\begin{lstlisting}
case class B(z: Int)  // Tuple with one part.
case class C()        // Tuple with no parts.
\end{lstlisting}

Scala has a special syntax for empty case class\index{empty case class}es:
\begin{lstlisting}
case object C // Similar to `case class C()`.
\end{lstlisting}
There are two main differences between \lstinline!case class C()!
and \lstinline!case object C!:
\begin{itemize}
\item A \lstinline!case object! cannot have type parameters, while we may
define, if needed, a \lstinline!case class C[X, Y, Z]()! with type
parameters \lstinline!X!, \lstinline!Y!, \lstinline!Z!.
\item A \lstinline!case object! is allocated in memory only once, while
new values of a \lstinline!case class C()! will be allocated in memory
each time \lstinline!C()! is evaluated.
\end{itemize}
Other than that, \lstinline!case class C()! and \lstinline!case object C!
have the same meaning: a named tuple with zero parts, which we may
also view as a ``named \lstinline!Unit!\index{unit type!named}''
type. In this book, I will not use \lstinline!case object!s because
\lstinline!case class!es are more general.

Let us summarize the correspondence between tuples and case classes:
\begin{center}
\begin{tabular}{|c|c|}
\hline 
\textbf{Tuples} & \textbf{Case classes}\tabularnewline
\hline 
\hline 
\lstinline!(123, "xyz"): Tuple2[Int, String]! & \lstinline!case class A(x: Int, y: String)!\tabularnewline
\hline 
\lstinline!(123,): Tuple1[Int]! & \lstinline!case class B(z: Int)!\tabularnewline
\hline 
\lstinline!(): Unit! & \lstinline!case class C()!\tabularnewline
\hline 
\end{tabular}
\par\end{center}

\subsection{Pattern matching for case classes}

Scala performs pattern matching in two situations:
\begin{itemize}
\item destructuring definition: \lstinline[mathescape=true]!val $pattern$ = ...!
\item \lstinline!case! expression: \lstinline[mathescape=true]!case $pattern$ => ...!
\end{itemize}
Case classes can be used in both situations. A destructuring definition
can be used in a function whose argument is of case class type \lstinline!BagOfSocks!:
\begin{lstlisting}
case class MySock(size: Double, color: String)
case class BagOfSocks(sock: MySock, count: Int)

def printBag(bag: BagOfSocks): String = {
  val BagOfSocks(MySock(size, color), count) = bag // Destructure the `bag`.
  s"bag has $count $color socks of size $size"
}

val bag = BagOfSocks(MySock(10.5, "white"), 6)

scala> printBag(bag)
res0: String = bag has 6 white socks of size 10.5
\end{lstlisting}

A \lstinline!case! expression will destructure a value and compute
a result:
\begin{lstlisting}
def fitsMe(bag: BagOfSocks): Boolean = bag match {
  case BagOfSocks(MySock(size, _), _) => (size >= 10.5 && size <= 11.0)
}
\end{lstlisting}
In the code of this function, we match the \lstinline!bag! value
against the pattern \lstinline!BagOfSocks(MySock(size, _), _)!. This
pattern will always match and will define \lstinline!size! as a pattern
variable of type \lstinline!Double!.

The syntax for pattern matching expressions with case classes is similar
to the syntax for pattern matching of tuples, except for the presence
of the \emph{names} of the case classes. For example, removing the
case class names from the pattern
\begin{lstlisting}
case BagOfSocks(MySock(size, _), _) => ...
\end{lstlisting}
we obtain the nested tuple pattern 
\begin{lstlisting}
case ((size, _), _) => ...
\end{lstlisting}
that could be used for values of type \lstinline!((Double, String), Int)!.
We see that within pattern matching expressions, case classes behave
exactly as tuple types with added names. 

Scala's ``case classes'' got their name from their use in \lstinline!case!
expressions. It is usually more convenient to use \lstinline!match!
/ \lstinline!case! expressions with case classes than to use destructuring.

\section{Disjunctive types}

\subsection{Motivation and first examples\label{subsec:Disjunctive-Motivation-and-first-examples}}

In many situations, it is useful to have several different shapes
of data within the same type. As a first example, suppose we are looking
for real roots of a quadratic equation $x^{2}+bx+c=0$. There are
three cases: no real roots, one real root, and two real roots. It
is convenient to have a type that represents ``the real roots of
a quadratic equation''; call it \lstinline!RootsOfQ!. Inside that
type, we distinguish between the three cases, but outside it looks
like a single type.

Another example is the binary search algorithm that looks for an integer
$x$ in a sorted array. Either the algorithm finds the index of $x$,
or it determines that the array does not contain $x$. It is convenient
if the algorithm could return a single value of a type (call it \lstinline!SearchResult!)
that represents \emph{either} an index at which $x$ is found, \emph{or}
the absence of an index.

More generally, we may have computations that \emph{either} return
a value \emph{or} generate an error and fail to produce a result.
It is then convenient to return a value of type (call it \lstinline!Result!)
that represents either a correct result or an error message. 

In certain computer games, one has different types of ``rooms'',
each room having certain properties depending on its type. Some rooms
are dangerous because of monsters, other rooms contain useful objects,
certain rooms allow you to finish the game, and so on. We want to
represent all the different kinds of rooms uniformly, as a type \lstinline!Room!,
so that a value of type \lstinline!Room! automatically stores the
correct properties in each case.

In all these situations, data comes in several mutually exclusive
shapes. This data can be represented by a single type if that type
is able to describe a mutually exclusive set of cases:
\begin{itemize}
\item \lstinline!RootsOfQ! must be either the empty tuple \lstinline!()!,
or \lstinline!Double!, or a tuple \lstinline!(Double, Double)!
\item \lstinline!SearchResult! must be either \lstinline!Int! or the empty
tuple \lstinline!()!
\item \lstinline!Result! must be either an \lstinline!Int! value or a
\lstinline!String! message
\end{itemize}
We see that the empty tuple, also known as the \lstinline!Unit! type,
is natural to use in these situations. It is also helpful to assign
names to each of the cases:
\begin{itemize}
\item \lstinline!RootsOfQ! is ``no roots'' with value \lstinline!()!,
or ``one root'' with value \lstinline!Double!, or ``two roots''
with value \lstinline!(Double, Double)!
\item \lstinline!SearchResult! is ``index'' with value \lstinline!Int!,
or ``not found'' with value \lstinline!()!
\item \lstinline!Result! is ``value'' of type \lstinline!Int! or ``error
message'' of type \lstinline!String!
\end{itemize}
Scala's case classes provide exactly what we need here \textendash{}
named tuples with zero, one, two and more parts, and so it is natural
to use case classes instead of tuples:
\begin{itemize}
\item \lstinline!RootsOfQ! is a value of type \lstinline!case class NoRoots()!,
or a value of type \lstinline!case class OneRoot(x: Double)!, or
a value of type \lstinline!case class TwoRoots(x: Double, y: Double)!
\item \lstinline!SearchResult! is a value of type \lstinline!case class Index(Int)!
or a value of type \lstinline!case class NotFound()!
\item \lstinline!Result! is a value of type \lstinline!case class Value(x: Int)!
or a value of type \lstinline!case class Error(message: String)!
\end{itemize}
Our three examples are now described as types that select one case
class out of a given set. It remains to see how Scala defines such
types. For instance, the definition of \lstinline!RootsOfQ! needs
to indicate that the case classes \lstinline!NoRoots!, \lstinline!OneRoot!,
and \lstinline!TwoRoots! are exactly the three alternatives described
by the type \lstinline!RootsOfQ!. The Scala syntax for that definition
looks like this:
\begin{lstlisting}
sealed trait RootsOfQ
final case class NoRoots()                      extends RootsOfQ
final case class OneRoot(x: Double)             extends RootsOfQ
final case class TwoRoots(x: Double, y: Double) extends RootsOfQ
\end{lstlisting}
In the definition of \lstinline!SearchResult!, we have two cases:
\begin{lstlisting}
sealed trait SearchResult
final case class Index(i: Int) extends SearchResult
final case class NotFound()    extends SearchResult
\end{lstlisting}
The definition of the \lstinline!Result! type is parameterized, so
that we can describe results of any type (while error messages are
always of type \lstinline!String!):
\begin{lstlisting}
sealed trait Result[A]
final case class Value[A](x: A)            extends Result[A]
final case class Error[A](message: String) extends Result[A]
\end{lstlisting}

The ``\lstinline!sealed trait! / \lstinline!final case class!''
syntax defines a type that represents a choice of one case class from
a fixed set of case classes. This kind of type is called a \textbf{\index{disjunctive type}disjunctive
type} in this book. 

\subsection{Solved examples: Pattern matching for disjunctive types\index{solved examples}}

Our first examples of disjunctive types are \lstinline!RootsOfQ!,
\lstinline!SearchResult!, and \lstinline!Result[A]! defined in the
previous section. We will now look at the Scala syntax for creating
values of disjunctive types and for using the created values.

Consider the disjunctive type \lstinline!RootsOfQ! having three case
classes (\lstinline!NoRoots!, \lstinline!OneRoot!, \lstinline!TwoRoots!).
The only way of creating a value of type \lstinline!RootsOfQ! is
to create a value of one of these case classes. This is done by writing
expressions such as \lstinline!NoRoots()!, \lstinline!OneRoot(2.0)!,
or \lstinline!TwoRoots(1.0, -1.0)!. Scala will accept these expressions
as having the type \lstinline!RootsOfQ!:
\begin{lstlisting}
scala> val x: RootsOfQ = OneRoot(2.0)
x: RootsOfQ = OneRoot(2.0)
\end{lstlisting}

Given a value \lstinline!x:RootsOfQ!, how can we use it, say, as
a function argument? The main tool for working with values of disjunctive
types is pattern matching with \lstinline!match! / \lstinline!case!
expressions. In Chapter~\ref{chap:2-Mathematical-induction}, we
used pattern matching to destructure tuples with syntax such as \lstinline!{ case (x, y) => ... }!.
To use \lstinline!match! / \lstinline!case! expressions with disjunctive
types, we may have to write \emph{more than one} \lstinline!case!
pattern in a \lstinline!match! expression, because we need to match
several possible cases of the disjunctive type:
\begin{lstlisting}
def f(r: RootsOfQ): String = r match {
  case NoRoots()       => "no real roots"
  case OneRoot(r)      => s"one real root: $r"
  case TwoRoots(x, y)  => s"real roots: ($x, $y)"
}

scala> f(x)
res0: String = "one real root: 2.0"
\end{lstlisting}
If we only need to recognize a specific case of a disjunctive type,
we can match all other cases with an underscore:
\begin{lstlisting}
scala> x match {
  case OneRoot(r)   => s"one real root: $r"
  case _            => "have something else"
}
res1: String = one real root: 2.0
\end{lstlisting}
The \lstinline!match! / \lstinline!case! expression represents a
choice over possible values of a given type. Note the similarity with
this code:

\begin{lstlisting}
def f(x: Int): Int = x match {
  case 0    => println(s"error: must be nonzero"); -1
  case 1    => println(s"error: must be greater than 1"); -1
  case _    => x
}
\end{lstlisting}
The values \lstinline!0! and \lstinline!1! are some possible values
of type \lstinline!Int!, just as \lstinline!OneRoot(1.0)! is a possible
value of type \lstinline!RootsOfQ!. When used with disjunctive types,
\lstinline!match! / \lstinline!case! expressions will usually contain
a complete list of possibilities. If the list of cases is incomplete,
the Scala compiler will print a warning:
\begin{lstlisting}
scala> def g(x: RootsOfQ): String = x match {
         case OneRoot(r) => s"one real root: $r"
      }
<console>:14: warning: match may not be exhaustive.
It would fail on the following inputs: NoRoots(), TwoRoots(_, _)
       def g(x: RootsOfQ): String = x match {
                                    ^
\end{lstlisting}
This code defines a \index{partial function}\emph{partial} function
\lstinline!g! that can be applied only to values of the form \lstinline!OneRoot(...)!
and will fail for other values.

Let us look at more examples of using the disjunctive types we just
defined.

\subsubsection{Example \label{subsec:disj-Example-rootsofq-1}\ref{subsec:disj-Example-rootsofq-1}}

Given a sequence of quadratic equations, compute a sequence containing
their real roots as values of type \lstinline!RootsOfQ!.

\subparagraph{Solution}

Define a case class representing a quadratic equation $x^{2}+bx+c=0$:
\begin{lstlisting}
case class QEqu(b: Double, c: Double)
\end{lstlisting}
The following function determines how many real roots an equation
has:
\begin{lstlisting}
def solve(quadraticEqu: QEqu): RootsOfQ = {
   val QEqu(b, c) = quadraticEqu // Destructure QEqu.
   val d = b * b / 4 - c
   if (d > 0) {
     val s = math.sqrt(d)
     TwoRoots(b / 2 - s, b / 2 + s)
   } else if (d == 0.0) OneRoot(b / 2)
   else NoRoots()
}
\end{lstlisting}
Test the \lstinline!solve! function:
\begin{lstlisting}
scala> solve(QEqu(1,1))
res1: RootsOfQ = NoRoots()

scala> solve(QEqu(1,-1))
res2: RootsOfQ = TwoRoots(-0.6180339887498949,1.618033988749895) 

scala> solve(QEqu(6,9))
res3: RootsOfQ = OneRoot(3.0) 
\end{lstlisting}
We can now implement the function \lstinline!findRoots!,
\begin{lstlisting}
def findRoots(equs: Seq[QEqu]): Seq[RootsOfQ] = equs.map(solve)
\end{lstlisting}
If the function \lstinline!solve! will not be used often, we may
want to write it inline as a nameless function:
\begin{lstlisting}
def findRoots(equs: Seq[QEqu]): Seq[RootsOfQ] = equs.map { case QEqu(b, c) =>
  (b * b / 4 - c) match {
    case d if d > 0   =>
      val s = math.sqrt(d)
      TwoRoots(b / 2 - s, b / 2 + s)
    case 0.0          => OneRoot(b / 2)
    case _            => NoRoots()
  }
}
\end{lstlisting}
This code depends on some features of Scala syntax. We can use the
partial function \lstinline!{ case QEqu(b, c) => ... }! directly
as the argument of \lstinline!.map! instead of defining this function
separately. This avoids having to destructure \lstinline!QEqu! at
a separate step. The \lstinline!if! / \lstinline!else! expression
is replaced by an ``embedded''\index{embedded `if`} \lstinline!if!
within the \lstinline!case! expression, which is easier to read.
Test the final code:
\begin{lstlisting}
scala> findRoots(Seq(QEqu(1,1), QEqu(2,1)))
res4: Seq[RootsOfQ] = List(NoRoots(), OneRoot(1.0)) 
\end{lstlisting}


\subsubsection{Example \label{subsec:disj-Example-rootsofq}\ref{subsec:disj-Example-rootsofq}}

Given a sequence of values of type \lstinline!RootsOfQ!, compute
a sequence containing only the single roots. Example test:
\begin{lstlisting}
def singleRoots(rs: Seq[RootsOfQ]): Seq[Double] = ???

scala> singleRoots(Seq(TwoRoots(-1, 1), OneRoot(3.0), OneRoot(1.0), NoRoots()))
res5: Seq[Double] = List(3.0, 1.0) 
\end{lstlisting}


\subparagraph{Solution}

We apply \lstinline!.filter! and \lstinline!.map! to the sequence
of roots:
\begin{lstlisting}
def singleRoots(rs: Seq[RootsOfQ]): Seq[Double] = rs.filter {
  case OneRoot(x) => true
  case _          => false
}.map { case OneRoot(x) => x }
\end{lstlisting}
In the \lstinline!.map! operation, we need to cover only the one-root
case because the other two possibilities have been ``filtered out''
by the preceding \lstinline!.filter! operation.

\subsubsection{Example \label{subsec:disj-Example-searchresult}\ref{subsec:disj-Example-searchresult}}

Implement binary search returning a \lstinline!SearchResult!. We
will modify the binary search implementation from Example~\ref{subsec:ch2Example-binary-search-seq-4}(b)
so that it returns a \lstinline!NotFound! value when appropriate.

\subparagraph{Solution}

The code from Example~\ref{subsec:ch2Example-binary-search-seq-4}(b)
will return \emph{some} index even if the given number is not present
in the array: 
\begin{lstlisting}
scala> binSearch(Array(1, 3, 5, 7), goal = 5)
res6: Int = 2

scala> binSearch(Array(1, 3, 5, 7), goal = 4)
res7: Int = 1
\end{lstlisting}
When the number is not present, the array's element at the computed
index will not be equal to \lstinline!goal!. We should return \lstinline!NotFound()!
in that case. The new code can be written as a \lstinline!match!
/ \lstinline!case! expression for clarity:
\begin{lstlisting}
def safeBinSearch(xs: Seq[Int], goal: Int): SearchResult =
  binSearch(xs, goal) match {
    case n if xs(n) == goal   => Index(n) 
    case _                    => NotFound()
  }
\end{lstlisting}
To test:
\begin{lstlisting}
scala> safeBinSearch(Array(1, 3, 5, 7), 5)
res8: SearchResult = Index(2)

scala> safeBinSearch(Array(1, 3, 5, 7), 4)
res9: SearchResult = NotFound()
\end{lstlisting}


\subsubsection{Example \label{subsec:disj-Example-resultA}\ref{subsec:disj-Example-resultA}}

Use the disjunctive type \lstinline!Result[Int]! to implement ``safe
integer arithmetic'', where a division by zero or a square root of
a negative number will give an error message. Define arithmetic operations
directly for values of type \lstinline!Result[Int]!. When errors
occur, abandon further computations.

\subparagraph{Solution}

Begin by implementing the square root:
\begin{lstlisting}
def sqrt(r: Result[Int]): Result[Int] = r match {
  case Value(x) if x >= 0  => Value(math.sqrt(x).toInt)
  case Value(x)            => Error(s"error: sqrt($x)")
  case Error(m)            => Error(m) // Keep the error message.
}
\end{lstlisting}
The square root is computed only if we have the \lstinline!Value(x)!
case, and only if $x\geq0$. If the argument \lstinline!r! was already
an \lstinline!Error! case, we keep the error message and perform
no further computations.

To implement the addition operation, we need a bit more work:
\begin{lstlisting}
def add(rx: Result[Int], ry: Result[Int]): Result[Int] = (rx, ry) match {
  case (Value(x), Value(y)) => Value(x + y)
  case (Error(m), _)        => Error(m) // Keep the error message.
  case (_, Error(m))        => Error(m)
}
\end{lstlisting}
This code illustrates nested patterns that match the tuple \lstinline!(rx, ry)!
against various possibilities. In this way, the code is clearer than
code written with nested \lstinline!if! / \lstinline!else! expressions.

Implementing the multiplication operation results in almost the same
code:
\begin{lstlisting}
def mul(rx: Result[Int], ry: Result[Int]): Result[Int] = (rx, ry) match {
  case (Value(x), Value(y)) => Value(x * y)
  case (Error(m), _)        => Error(m)
  case (_, Error(m))        => Error(m)
}
\end{lstlisting}
To avoid repetition, we may define a general function that ``lifts''\index{lifting}
operations on integers to operations on \lstinline!Result[Int]! types:
\begin{lstlisting}
def do2(rx: Result[Int], ry: Result[Int])(op: (Int, Int) => Int): Result[Int] =
  (rx, ry) match {
    case (Value(x), Value(y)) => Value(op(x, y))
    case (Error(m), _)        => Error(m)
    case (_, Error(m))        => Error(m)
  }
\end{lstlisting}
Now we can easily ``lift'' any binary operation on integers to a
binary operation on \lstinline!Result[Int]!, assuming that the operation
never generates an error:
\begin{lstlisting}
def sub(rx: Result[Int], ry: Result[Int]): Result[Int] = do2(rx, ry){ (x, y) => x - y }
\end{lstlisting}
 Custom code is still needed for operations that \emph{may} generate
errors:
\begin{lstlisting}
def div(rx: Result[Int], ry: Result[Int]): Result[Int] = (rx, ry) match {
  case (Value(x), Value(y)) if y != 0  => Value(x / y)
  case (Value(x), Value(y))            => Error(s"error: $x / $y")
  case (Error(m), _)                   => Error(m)
  case (_, Error(m))                   => Error(m)
}
\end{lstlisting}
We can now test the new ``safe arithmetic'' on simple calculations:
\begin{lstlisting}
scala> add(Value(1), Value(2))
res10: Result[Int] = Value(3)

scala> div(add(Value(1), Value(2)), Value(0))
res11: Result[Int] = Error(error: 3 / 0)
\end{lstlisting}
We see that indeed all further computations are abandoned once an
error occurs. An error message shows only the immediate calculation
that generated the error. For instance, the error message for $20+1/0$
never mentions $20$:
\begin{lstlisting}
scala> add(Value(20), div(Value(1), Value(0)))
res12: Result[Int] = Error(error: 1 / 0)

scala> add(sqrt(Value(-1)), Value(10))
res13: Result[Int] = Error(error: sqrt(-1))
\end{lstlisting}


\subsection{Standard disjunctive types: \texttt{Option}, \texttt{Either}, \texttt{Try}}

The Scala library defines the disjunctive types \lstinline!Option!,
\lstinline!Either!, and \lstinline!Try! because they are used often.
We now look at each of them in turn.

\paragraph{The \texttt{Option} type}

is a disjunctive type with two cases: the empty tuple and a one-element
tuple. The names of the two case classes are \lstinline!None! and
\lstinline!Some!. If the \lstinline!Option! type were not already
defined in the standard library, one could define it with the code
\begin{lstlisting}
sealed trait Option[T]
final case object None         extends Option[Nothing]
final case class Some[T](t: T) extends Option[T]
\end{lstlisting}
This code is similar to the type \lstinline!SearchResult! defined
in Section~\ref{subsec:Disjunctive-Motivation-and-first-examples},
except that \lstinline!Option! has a type parameter instead of a
fixed type \lstinline!Int!. Another difference is the use of a \lstinline!case object!
for the empty case instead of an empty case class, such as \lstinline!None()!.
Since Scala's \lstinline!case object!s cannot have type parameters,
the type parameter in the definition of \lstinline!None! must be
set to the special type \lstinline!Nothing!, which is a type with
\emph{no} values (also called the \textbf{void} \textbf{type}\index{void type}).

An alternative (implemented in libraries such as \texttt{scalaz})
is to define the empty option value as
\begin{lstlisting}
final case class None[T]() extends Option[T]
\end{lstlisting}
In that implementation, the empty option \lstinline!None[T]()! has
a type parameter.

Several consequences follow from the Scala library's decision to define
\lstinline!None! without a type parameter. One consequence is that
\lstinline!None! can be reused as a value of type \lstinline!Option[A]!
for any type \lstinline!A!:
\begin{lstlisting}
scala> val y: Option[Int] = None
y: Option[Int] = None

scala> val z: Option[String] = None
z: Option[String] = None
\end{lstlisting}

Typically, \lstinline!Option! is used in situations where a value
may be either present or missing, especially when a missing value
is \emph{not an error}. The missing-value case is represented by \lstinline!None!,
while \lstinline!Some(x)! means that a value \lstinline!x! is present.

\subsubsection{Example \label{subsec:Disjunctive-Example-option-1}\ref{subsec:Disjunctive-Example-option-1}\index{solved examples}}

Suppose that information about subscribers to a certain online service
must contain a name and an email address, but a telephone number is
optional. To represent this information, we may define a case class
like this,
\begin{lstlisting}
case class Subscriber(name: String, email: String, phone: Option[Long])
\end{lstlisting}
What if we represent the missing telephone number by a special value
such as \lstinline!-1! and use the simpler type \lstinline!Long!
instead of \lstinline!Option[Long]!? The disadvantage is that we
would need to \emph{remember} to check for the special value \lstinline!-1!
in all functions that take the telephone number as an argument. Looking
at a function such as \lstinline!sendSMS(phone: Long)! at a different
place in the code, a programmer might forget that the telephone number
is actually optional. In contrast, the type signature \lstinline!sendSMS(phone: Option[Long])!
unambiguously indicates that the telephone number might be missing
and helps the programmer to remember to handle both cases.

Pattern-matching code involving \lstinline!Option! can handle the
two cases like this:
\begin{lstlisting}
def getDigits(phone: Option[Long]): Option[Seq[Long]] = phone match {
  case None               => None       // Do nothing.
  case Some(number)       => Some(digitsOf(number))
}
\end{lstlisting}
Here we used the function \lstinline!digitsOf! defined in Section~\ref{sec:ch2Converting-a-single}. 

At the two sides of \lstinline!case None => None!, the value \lstinline!None!
has different types, namely \lstinline!Option[Long]! and \lstinline!Option[Seq[Long]]!.
Since these types are declared in the type signature of the function
\lstinline!getDigits!, the Scala compiler is able to figure out the
types of all expressions in the \lstinline!match! / \lstinline!case!
construction. So, pattern-matching code can be written without explicit
type annotations\index{type annotation} such as \lstinline!(None: Option[Long])!.

If we now need to compute the number of digits, we can write
\begin{lstlisting}
def numberOfDigits(phone: Option[Long]): Option[Long] = getDigits(phone) match {
  case None               => None       // Do nothing.
  case Some(digits)       => Some(digits.length)
}
\end{lstlisting}

These examples perform a computation when an \lstinline!Option! value
is non-empty, and leave it empty otherwise. This design pattern is
used often. To avoid repeating the code, we can implement this design
pattern as a function that takes the computation as an argument \lstinline!f!:
\begin{lstlisting}
def doComputation(x: Option[Long], f: Long => Long): Option[Long] = x match {
  case None               => None       // Do nothing.
  case Some(i)            => Some(f(i))
}
\end{lstlisting}
It is then natural to generalize this function to arbitrary types
using type parameters instead of a fixed type \lstinline!Long!. The
resulting function is usually called \lstinline!fmap!:
\begin{lstlisting}
def fmap[A, B](f: A => B): Option[A] => Option[B] = {
  case None               => None       // Do nothing.
  case Some(a)            => Some(f(a))
}

scala> fmap(digitsOf)(Some(4096))
res0: Option[Seq[Long]] = Some(List(4, 0, 9, 6))

scala> fmap(digitsOf)(None)
res1: Option[Seq[Long]] = None
\end{lstlisting}
We say that the \lstinline!fmap! operation \textbf{lifts}\index{lifting}
a given function of type \lstinline!A => B! to the type \lstinline!Option[A] => Option[B]!. 

It is important to keep in mind that the code \lstinline!case Some(a) => Some(f(a))!
changes the type of the option value. On the left side of the arrow,
the type is \lstinline!Option[A]!, while on the right side it is
\lstinline!Option[B]!. The Scala compiler knows this from the given
type signature of \lstinline!fmap!, so an explicit type parameter,
\lstinline!Some[B](f(a))!, is not needed.

The Scala library implements an equivalent function as a method on
the \lstinline!Option! class, with the syntax \lstinline!x.map(f)!
rather than \lstinline!fmap(f)(x)!. We can concisely rewrite the
previous code using the standard library methods as
\begin{lstlisting}
def getDigits(phone: Option[Long]): Option[Seq[Long]] = phone.map(digitsOf)
def numberOfDigits(phone: Option[Long]): Option[Long] = phone.map(digitsOf).map(_.length)
\end{lstlisting}
We see that the \lstinline!.map! operation for the \lstinline!Option!
type is analogous to the \lstinline!.map! operation for sequences. 

The similarity between \lstinline!Option[A]! and \lstinline!Seq[A]!
is clearer if we view \lstinline!Option[A]! as a special kind of
``sequence'' whose length is restricted to be either $0$ or $1$.
So, \lstinline!Option[A]! can have all the operations of \lstinline!Seq[A]!
except operations such as \lstinline!.concat! that may grow the sequence
beyond length $1$. The standard operations defined on \lstinline!Option!
include \lstinline!.map!, \lstinline!.filter!, \lstinline!.forall!,
\lstinline!.exists!, \lstinline!.flatMap!, and \lstinline!.foldLeft!.

\subsubsection{Example \label{subsec:Disjunction-Example-Option-flatMap}\ref{subsec:Disjunction-Example-Option-flatMap}}

Given a phone number as \lstinline!Option[Long]!, extract the country
code if it is present. (Assume that the country code is any digits
in front of the $10$-digit number; for the phone number $18004151212$,
the country code is $1$.) The result must be again of type \lstinline!Option[Long]!.

\subparagraph{Solution}

If the phone number is a positive integer $n$, we may compute the
country code simply as \lstinline!n/10000000000L!. However, if the
result of that division is zero, we should return an empty \lstinline!Option!
(i.e.~the value \lstinline!None!) rather than \lstinline!0!. To
implement this logic, we may begin by writing this code,
\begin{lstlisting}
def countryCode(phone: Option[Long]): Option[Long] = phone match {
  case None      => None
  case Some(n)   =>
    val countryCode = n / 10000000000L
    if (countryCode != 0L) Some(countryCode) else None 
}
\end{lstlisting}
We may notice that we have reimplemented the design pattern similar
to \lstinline!.map! in this code, namely ``if \lstinline!None!
then return \lstinline!None!, else do a computation''. So we may
try to rewrite the code as
\begin{lstlisting}
def countryCode(phone: Option[Long]): Option[Long] = phone.map { n =>
    val countryCode = n / 10000000000L
    if (countryCode != 0L) Some(countryCode) else None 
} // Type error: the result is Option[Option[Long]], not Option[Long].
\end{lstlisting}
This code does not compile: we are returning an \lstinline!Option[Long]!
within a function lifted via \lstinline!.map!, so the resulting type
is \lstinline!Option[Option[Long]]!. We may use \lstinline!.flatten!
to convert \lstinline!Option[Option[Long]]! to the required type
\lstinline!Option[Long]!,
\begin{lstlisting}
def countryCode(phone: Option[Long]): Option[Long] = phone.map { n =>
    val countryCode = n / 10000000000L
    if (countryCode != 0L) Some(countryCode) else None 
}.flatten // Types are correct now.
\end{lstlisting}
Since the \lstinline!.flatten! follows a \lstinline!.map!, we can
rewrite the code using \lstinline!.flatMap!:
\begin{lstlisting}
def countryCode(phone: Option[Long]): Option[Long] = phone.flatMap { n =>
    val countryCode = n / 10000000000L
    if (countryCode != 0L) Some(countryCode) else None 
} // Types are correct now.
\end{lstlisting}
Another way of implementing this example is to notice the design pattern
``if condition does not hold, return \lstinline!None!, otherwise
keep the value''. For an \lstinline!Option! type, this is equivalent
to the \lstinline!.filter! operation (recall that \lstinline!.filter!
returns an empty sequence if the predicate never holds). The code
is
\begin{lstlisting}
def countryCode(phone: Option[Long]): Option[Long] = phone.map(_ / 10000000000L).filter(_ != 0L)
\end{lstlisting}
Test it:
\begin{lstlisting}
scala> countryCode(Some(18004151212L))
res0: Option[Long] = Some(1)

scala> countryCode(Some(8004151212L))
res1: Option[Long] = None
\end{lstlisting}


\subsubsection{Example \label{subsec:Disjunction-Example-Option-getOrElse}\ref{subsec:Disjunction-Example-Option-getOrElse}}

Add a new requirement to Example~\ref{subsec:Disjunction-Example-Option-flatMap}:
if the country code is not present, we should return the default country
code $1$.

\subparagraph{Solution}

This is an often used design pattern: ``if empty, substitute a default
value''. The Scala library has the method \lstinline!.getOrElse!
for this purpose:
\begin{lstlisting}
scala> Some(100).getOrElse(1)
res2: Int = 100

scala> None.getOrElse(1)
res3: Int = 1
\end{lstlisting}
So we can implement the new requirement as
\begin{lstlisting}
scala> countryCode(Some(8004151212L)).getOrElse(1L)
res4: Long = 1
\end{lstlisting}


\paragraph{Using \texttt{Option} with collections}

Many Scala library methods return an \lstinline!Option! as a result.
The main examples are \lstinline!.find!, \lstinline!.headOption!,
and \lstinline!.lift! for sequences, and \lstinline!.get! for dictionaries.

The \lstinline!.find! method returns the first element satisfying
a predicate:
\begin{lstlisting}
scala> (1 to 10).find(_ > 5)
res0: Option[Int] = Some(6)

scala> (1 to 10).find(_ > 10) // No element is > 10.
res1: Option[Int] = None
\end{lstlisting}

The \lstinline!.lift! method returns the element of a sequence at
a given index:
\begin{lstlisting}
scala> (10 to 100).lift(0)
res2: Option[Int] = Some(10)

scala> (10 to 100).lift(1000) // No element at index 1000.
res3: Option[Int] = None
\end{lstlisting}

The \lstinline!.headOption! method returns the first element of a
sequence, unless the sequence is empty. This is equivalent to \lstinline!.lift(0)!:
\begin{lstlisting}
scala> Seq(1,2,3).headOption
res4: Option[Int] = Some(1)

scala> Seq(1,2,3).filter(_ > 10).headOption
res5: Option[Int] = None
\end{lstlisting}
Applying \lstinline!.find(p)! computes the same result as \lstinline!.filter(p).headOption!,
but \lstinline!.find(p)! may be faster.

The \lstinline!.get! method for a dictionary returns the value if
it exists for a given key, and returns \lstinline!None! if the key
is not in the dictionary:
\begin{lstlisting}
scala> Map(10 -> "a", 20 -> "b").get(10)
res6: Option[String] = Some(a)

scala> Map(10 -> "a", 20 -> "b").get(30)
res7: Option[String] = None 
\end{lstlisting}
The \lstinline!.get! method provides safe by-key access to dictionaries,
unlike the direct access method that may fail:
\begin{lstlisting}
scala> Map(10 -> "a", 20 -> "b")(10)
res8: String = a 

scala> Map(10 -> "a", 20 -> "b")(30)
java.util.NoSuchElementException: key not found: 30
  at scala.collection.MapLike$class.default(MapLike.scala:228)
  at scala.collection.AbstractMap.default(Map.scala:59)
  ... 32 elided
\end{lstlisting}
Similarly, \lstinline!.lift! provides safe by-index access to collections,
unlike the direct access that may fail:
\begin{lstlisting}
scala> Seq(10,20,30)(0)
res9: Int = 10

scala> Seq(10,20,30)(5)
java.lang.IndexOutOfBoundsException: 5
  at scala.collection.LinearSeqOptimized$class.apply(LinearSeqOptimized.scala:65)
  at scala.collection.immutable.List.apply(List.scala:84)
  ... 32 elided
\end{lstlisting}


\paragraph{The \texttt{Either} type}

The standard disjunctive type \lstinline!Either[A, B]! has two type
parameters and is often used for computations that report errors.
By convention, the \emph{first} type (\lstinline!A!) is the type
of error, and the \emph{second} type (\lstinline!B!) is the type
of the (non-error) result. The names of the two cases are \lstinline!Left!
and \lstinline!Right!. A possible definition of \lstinline!Either!
may be written as
\begin{lstlisting}
sealed trait Either[A, B]
final case class  Left[A, B](value: A) extends Either[A, B]
final case class Right[A, B](value: B) extends Either[A, B]
\end{lstlisting}
By convention, a value \lstinline!Left(x)! represents an error, and
a value \lstinline!Right(y)! represents a valid result.

As an example, the following function substitutes a default value
and logs the error information:
\begin{lstlisting}
def logError(x: Either[String, Int], default: Int): Int = x match {
  case Left(error)  => println(s"Got error: $error"); default
  case Right(res)   => res
}
\end{lstlisting}
To test:
\begin{lstlisting}
scala> logError(Right(123), -1)
res1: Int = 123

scala> logError(Left("bad result"), -1)
Got error: bad result
res2: Int = -1
\end{lstlisting}

Why use \lstinline!Either! instead of \lstinline!Option! for computations
that may fail? A failing computation such as \lstinline!"xyz".toInt!
cannot return a result, and sometimes we might use \lstinline!None!
to indicate that a result is not available. However, when the result
is a requirement for further calculations, we will usually need to
know exactly \emph{which} error prevented the result from being available.
The \lstinline!Either! type may provide detailed information about
such errors, which \lstinline!Option! cannot do. 

The \lstinline!Either! type generalizes the type \lstinline!Result!
defined in Section~\ref{subsec:Disjunctive-Motivation-and-first-examples}
to an arbitrary error type instead of \lstinline!String!. We have
seen its usage in Example~\ref{subsec:disj-Example-resultA}, where
the design pattern was ``if value is present, do a computation, otherwise
keep the error''. This design pattern is implemented by the \lstinline!.map!
method on \lstinline!Either!:
\begin{lstlisting}[numbers=left,numberstyle={\small}]
scala> Right(1).map(_ + 1)
res0: Either[Nothing, Int] = Right(2)

scala> Left[String, Int]("error").map(_ + 1)
res1: Either[String, Int] = Left("error")
\end{lstlisting}
The type \lstinline!Nothing! was filled in by the Scala compiler
because we did not specify the first type parameter of \lstinline!Right!
in line 1.

The methods \lstinline!.filter!, \lstinline!.flatMap!, \lstinline!.fold!,
and \lstinline!.getOrElse! are also defined for \lstinline!Either!,
with the same convention that a \lstinline!Left! value represents
an error.\footnote{These methods are available in Scala 2.12 or a later version.}

\paragraph{Exceptions and the \texttt{Try} type}

When computations fail for any reason, Scala generates an \textbf{exception\index{exception}}
instead of returning a value. An exception means that the evaluation
of some expression was stopped without returning a result.

As an example, exceptions are generated when the available memory
is too small to store the resulting data (as we saw in Section~\ref{subsec:Lazy-values-iterators-and-streams}),
or if a stack overflow occurs during the computation (as we saw in
Section~\ref{subsec:Tail-recursion}). Exceptions may also occur
due to programmer's error: when a pattern matching operation fails,
when a requested key does not exist in a dictionary, or when the \lstinline!.head!
operation is applied to an empty list.

Motivated by these examples, we may distinguish ``planned\index{planned exception}''
and ``unplanned'' exceptions. 

A \textbf{planned} exception is generated by programmer's code via
the \lstinline!throw! syntax:
\begin{lstlisting}
scala> throw new Exception("This is a test... this is only a test.")
java.lang.Exception: This is a test... this is only a test.
\end{lstlisting}
The Scala library contains a \lstinline!throw! operation in various
places, such as in the code for applying the \lstinline!.head! method
to an empty sequence, as well as in other situations where exceptions
are generated due to programmer's errors. These exceptions are generated
deliberately and in well-defined situations. Although these exceptions
indicate errors, these errors are anticipated in advance and so may
be handled by the programmer.

For example, many Java libraries will generate exceptions when function
arguments have unexpected values, when a network operation takes too
long or a network connection is unexpectedly broken, when a file is
not found or cannot be read due to access permissions, and in many
other situations. All these exceptions are ``planned'' because they
are generated explicitly by library code such as \lstinline!throw new FileNotFoundException(...)!.
The programmer's code is expected to catch these exceptions, to handle
the error, and to continue the evaluation of the program.

An \textbf{unplanned} exception\index{unplanned exception} is generated
by the Java runtime system when critical errors occur, such as an
out-of-memory error. It is rare that a programmer writes \lstinline!val y = f(x)!
while \emph{expecting} that an out-of-memory exception will sometimes
occur at that point.\footnote{Just once in the author's experience, an out-of-memory condition had
to be anticipated in an Android app.} An unplanned exception indicates a serious and unforeseen problem
with memory or another critically important resource, such as the
operating system's threads or file handles. Such problems usually
cannot be fixed and will prevent the program from running any further.
It is reasonable that the program evaluation should abruptly stop
(or ``crash'' as programmers say) after such an error.

The use of planned exceptions assumes that the programmer will write
code to handle each exception. This assumption makes it significantly
harder to write programs correctly: it is hard to figure out and to
keep in mind all the possible exceptions that a given library function
may \lstinline!throw! in its code and in the code of all other libraries
on which it depends. Instead of using exceptions for indicating errors,
Scala programmers can write functions that return a disjunctive type
such as \lstinline!Either!, describing both possibilities (a correct
result or an error condition). Users of these functions will \emph{have}
to do pattern matching on the result values. This helps programmers
to remember and to handle all relevant error situations.

However, programmers will often need to use Java or Scala libraries
that \lstinline!throw! exceptions. To help write code for these situations,
the Scala library contains a helper function called \lstinline!Try()!
and a disjunctive type also called \lstinline!Try!. The type \lstinline!Try[A]!
can be seen as similar to \lstinline!Either[Throwable, A]!, where
\lstinline!Throwable! is the general type of all exceptions (i.e.~values
to which a \lstinline!throw! operation can be applied). The two parts
of the disjunctive type \lstinline!Try[A]! are called \lstinline!Failure!
and \lstinline!Success[A]! (instead of \lstinline!Left[Throwable]!
and \lstinline!Right[A]! in the \lstinline!Either! type). The function
\lstinline!Try(expr)! will catch all exceptions thrown while the
expression \lstinline!expr! is evaluated. If the evaluation of \lstinline!expr!
succeeds and returns a value \lstinline!x:A!, the value of \lstinline!Try(expr)!
will be \lstinline!Success(x)!. Otherwise it will be \lstinline!Failure(t)!,
where \lstinline!t:Throwable! is the value associated with the generated
exception. Here is an example of using \lstinline!Try!:
\begin{lstlisting}
import scala.util.{Try,Success,Failure}

scala> Try("xyz".toInt)
res0: Try[Int] = Failure(java.lang.NumberFormatException: For input string: "xyz")

scala> Try("0002".toInt)
res1: Try[Int] = Success(2) 
\end{lstlisting}
The code \lstinline!Try("xyz".toInt)! does not generate any exceptions
and will not crash the program. Any computation that may \lstinline!throw!
an exception can be enclosed in a \lstinline!Try()!, and the exception
will be caught and encapsulated within the disjunctive type as a \lstinline!Failure(...)!
value.

The methods \lstinline!.map!, \lstinline!.filter!, \lstinline!.flatMap!,
\lstinline!.foldLeft! are defined for the \lstinline!Try! class
similarly to the \lstinline!Either! type. One additional feature
of \lstinline!Try! is to catch exceptions generated by the function
arguments of \lstinline!.map!, \lstinline!.filter!, \lstinline!.flatMap!,
and other standard methods:

\begin{wrapfigure}{l}{0.58\columnwidth}%
\vspace{-0.9\baselineskip}

\begin{lstlisting}
scala> val y = x.map(y => throw new Exception("ouch"))
y: Try[Int] = Failure(java.lang.Exception: ouch)

scala> val z = x.filter(y => throw new Exception("huh"))
z: Try[Int] = Failure(java.lang.Exception: huh)
\end{lstlisting}
\vspace{-0.9\baselineskip}
\end{wrapfigure}%

\noindent In this example, the values \lstinline!y! and \lstinline!z!
were computed \emph{successfully} even though exceptions were thrown
while the function arguments of \lstinline!.map! and \lstinline!.filter!
were evaluated. Other code can use pattern matching on the values
\lstinline!y! and \lstinline!z! and examine those exceptions. However,
it is important that these exceptions were caught and the program
did not crash, so the other code is \emph{able} to run. 

While the standard types \lstinline!Try! and \lstinline!Either!
will cover many use cases, programmers can also define custom disjunctive
types in order to represent all the anticipated failures or errors
in the business logic of a particular application. Representing all
errors in the types helps assure that the program will not crash because
of an exception that we forgot to handle or did not even know about.

\section{Lists and trees: recursive disjunctive types\label{sec:Lists-and-trees:recursive-disjunctive-types}}

Consider this code defining a disjunctive type \lstinline!NInt!:

\begin{wrapfigure}{l}{0.58\columnwidth}%
\vspace{-0.75\baselineskip}

\begin{lstlisting}
sealed trait NInt
final case class N1(x: Int)  extends NInt
final case class N2(n: NInt) extends NInt
\end{lstlisting}
\vspace{-0.75\baselineskip}
\end{wrapfigure}%

\noindent The type \lstinline!NInt! has two disjunctive parts, \lstinline!N1!
and \lstinline!N2!. But the definition of the case class \lstinline!N1!
refers to the type \lstinline!NInt! as if it were already defined. 

A type whose definition uses that same type is called a \index{recursive type}\textbf{recursive
type}. The type \lstinline!NInt! is an example of a recursive disjunctive
type.

We might imagine defining a disjunctive type \lstinline!X! whose
parts recursively refer to the same type \lstinline!X! (and/or to
each other) in complicated ways. What would kind of data would be
represented by such a type \lstinline!X!, and in what situations
would \lstinline!X! be useful? In general, this question is not easy
to answer. For instance, the simple definition
\begin{lstlisting}
final case class Bad(x: Bad)
\end{lstlisting}
is useless: we can create a value of type \lstinline!Bad! only if
we already have a value of type \lstinline!Bad!. This is an example
of an infinite type recursion\index{infinite type recursion}. We
will never be able to create any values of type \lstinline!Bad!,
which means that the type \lstinline!Bad! is void\index{void type}
(has no values, like the the special type \lstinline!Nothing!).

Chapter~\ref{chap:Recursive-types} studies recursive types in more
detail. For now, we will look at the main examples of recursive disjunctive
types that are \emph{known} to be useful. These examples are lists
and trees.

\subsection{Lists}

A list of values of type \lstinline!A! is either empty, or has one
value of type \lstinline!A!, or two values of type \lstinline!A!,
etc. We can visualize the type \lstinline!List[A]! as a disjunctive
type defined by
\begin{lstlisting}
sealed trait List[A]
final case class List0[A]()             extends List[A]
final case class List1[A](x: A)         extends List[A]
final case class List2[A](x1: A, x2: A) extends List[A]
???              // Need an infinitely long definition.
\end{lstlisting}
However, this definition is not practical: we cannot define a separate
case class for \emph{each} possible length. Instead, we define the
type \lstinline!List[A]! via mathematical induction on the length
of the list:
\begin{itemize}
\item Base case: empty list, \lstinline!case class List0[A]()!.
\item Inductive step: given a list of a previously defined length, say \lstinline!List!$_{n-1}$,
define a new case class \lstinline!List!$_{n}$ describing a list
with one more element of type \lstinline!A!. So we could define \lstinline!List!$_{n}=\,$\lstinline!(List!$_{n-1}$\lstinline!, A)!.
\end{itemize}
Let us try to write this inductive definition as code:
\begin{lstlisting}
sealed trait ListI[A]         // Inductive definition of a list.
final case class List0[A]()                     extends ListI[A]
final case class List1[A](prev: List0[A], x: A) extends ListI[A]
final case class List2[A](prev: List1[A], x: A) extends ListI[A]
???                 // Still need an infinitely long definition.
\end{lstlisting}
To avoid writing an infinitely long type definition, we need to use
a trick. Notice that all definitions of \lstinline!List1!, \lstinline!List2!,
etc., have a similar form (while \lstinline!List0! is not similar).
We can replace all the definitions \lstinline!List1!, \lstinline!List2!,
etc., by a single definition if we use the type \lstinline!ListI[A]!
recursively inside the case class:
\begin{lstlisting}
sealed trait ListI[A]         // Inductive definition of a list.
final case class List0[A]()                     extends ListI[A]
final case class ListN[A](prev: ListI[A], x: A) extends ListI[A]
\end{lstlisting}
The type definition has become recursive. For this trick to work,
it is important to use \lstinline!ListI[A]! and not \lstinline!ListN[A]!
inside the definition \lstinline!ListN[A]!; or else we would have
created an infinite type recursion\index{infinite type recursion}
similar to \lstinline!case class Bad! shown previously.

Since we obtained the type definition of \lstinline!ListI! via a
trick, let us verify that the code actually defines the disjunctive
type we wanted. 

To create a value of type \lstinline!ListI[A]!, we must use one of
the two available case classes. Using the first case class, we may
create a value \lstinline!List0()!. Since this empty case class does
not contain any values of type \lstinline!A!, it effectively represents
an empty list (the base case of the induction). Using the second case
class, we may create a value \lstinline!ListN(prev, x)! where \lstinline!x!
is of type \lstinline!A! and \lstinline!prev! is some previously
constructed value of type \lstinline!ListI[A]!. This represents the
inductive step, because the case class \lstinline!ListN! is a named
tuple containing \lstinline!ListI[A]! and \lstinline!A!. Now, the
same consideration recursively applies to constructing the value \lstinline!prev!,
which must be either an empty list or a pair containing another list
and an element of type \lstinline!A!. The assumption that the value
\lstinline!prev:ListI[A]! is already constructed is equivalent to
the inductive assumption that we already have a list of a previously
defined length. So, we have verified that \lstinline!ListI[A]! implements
the inductive definition shown above.

Examples of values of type \lstinline!ListI! are the empty list \lstinline!List0()!,
a one-element list \lstinline!ListN(List0(), x)!, and a two-element
list \lstinline!ListN(ListN(List0(), x), y)!.

To illustrate writing pattern-matching code using this type, let us
implement the method \lstinline!headOption!:
\begin{lstlisting}
@tailrec def headOption[A]: ListI[A] => Option[A] = {
  case List0()               => None
  case ListN(List0(), x)     => Some(x)
  case ListN(prev, _)        => headOption(prev)
}
\end{lstlisting}

The Scala library already defines the type \lstinline!List[A]! in
an equivalent but different way: its case classes are named differently,
and the second case class (with the special name \lstinline!::!)
places the value of type \lstinline!A! \emph{before} the previously
constructed list,
\begin{lstlisting}
sealed trait List[A]
final case object Nil extends List[Nothing]
final case class ::[A](head: A, tail: List[A]) extends List[A]
\end{lstlisting}
Because ``operator-like'' case class names, such as \lstinline!::!,
support the infix syntax, we may write \lstinline!head :: tail! instead
of \lstinline!::(head, tail)!. Pattern matching with the standard
\lstinline!List! class looks like this:
\begin{lstlisting}
def headOption[A]: List[A] => Option[A] = {
  case Nil            => None
  case head :: tail   => Some(head)
}
\end{lstlisting}
Examples of values created using Scala's standard \lstinline!List!
type are the empty list \lstinline!Nil!, a one-element list \lstinline!x :: Nil!,
and a two-element list \lstinline!x :: y :: Nil!. The same syntax
\lstinline!x :: y :: Nil! is used both for creating values of type
\lstinline!List! and for pattern-matching on such values. 

The Scala library also defines the helper function \lstinline!List()!,
so that \lstinline!List()! is the same as \lstinline!Nil! and \lstinline!List(1, 2, 3)!
is the same as \lstinline!1 :: 2 :: 3 :: Nil!. Lists are easier to
use in the syntax \lstinline!List(1, 2, 3)!. Pattern matching can
also use that syntax when convenient:
\begin{lstlisting}
val x: List[Int] = List(1, 2, 3)

x match {
  case List(a)         => ...
  case List(a, b, c)   => ...
  case _               => ...
}
\end{lstlisting}


\subsection{Tail recursion with \texttt{List}\label{subsec:Tail-recursion-with-list}}

Because the \lstinline!List! type is defined by induction, it is
straightforward to implement iterative computations with the \lstinline!List!
type using recursion.

A first example is the \lstinline!map! function. We use reasoning
by induction in order to figure out the implementation of \lstinline!map!.
The required type signature is
\begin{lstlisting}
def map[A, B](xs: List[A])(f: A => B): List[B] = ???
\end{lstlisting}
The base case is an empty list, and we return again an empty list:
\begin{lstlisting}
def map[A, B](xs: List[A])(f: A => B): List[B] = xs match {
  case Nil => Nil
  ...
\end{lstlisting}
In the inductive step, we have a pair \lstinline!(head, tail)! in
the case class \lstinline!::!, with \lstinline!head:A! and \lstinline!tail:List[A]!.
The pair can be pattern-matched with the syntax \lstinline!head :: tail!.
The \lstinline!map! function should apply the argument \lstinline!f!
to the head value, which will give the first element of the resulting
list. The remaining elements are computed by the induction assumption,
i.e.~by a recursive call to \lstinline!map!:
\begin{lstlisting}
def map[A, B](xs: List[A])(f: A => B): List[B] = xs match {
  case Nil           => Nil
  case head :: tail  => f(head) :: map(tail)(f) // Not tail-recursive.
\end{lstlisting}
While this implementation is straightforward and concise, it is not
tail-recursive. This will be a problem for large enough lists.

Instead of implementing the often-used methods such as \lstinline!.map!
or \lstinline!.filter! one by one, let us implement \lstinline!foldLeft!
because most of the other methods can be expressed via \lstinline!foldLeft!. 

The required type signature is
\begin{lstlisting}
def foldLeft[A, R](xs: List[A])(init: R)(f: (R, A) => R): R = ???
\end{lstlisting}
Reasoning by induction, we start with the base case \lstinline!xs == Nil!,
where the only possibility is to return the value \lstinline!init!:
\begin{lstlisting}
def foldLeft[A, R](xs: List[A])(init: R)(f: (R, A) => R): R = xs match {
    case Nil            => init
    ...
\end{lstlisting}
The inductive step for \lstinline!foldLeft! says that, given the
values \lstinline!head:A! and \lstinline!tail:List[A]!, we need
to apply the updater function to the previous accumulator value. That
value is \lstinline!init!. So we apply \lstinline!foldLeft! recursively
to the tail of the list once we have the updated accumulator value:
\begin{lstlisting}
@tailrec def foldLeft[A, R](xs: List[A])(init: R)(f: (R, A) => R): R =
  xs match {
    case Nil            => init
    case head :: tail   => 
      val newInit = f(init, head) // Update the accumulator.
      foldLeft(tail)(newInit)(f)  // Recursive call to `foldLeft`.
  }
\end{lstlisting}
This implementation is tail-recursive because the recursive call to
\lstinline!foldLeft! is the last expression returned in a \lstinline!case!
branch.

Another example is a function for reversing a list. The Scala library
defines the \lstinline!.reverse! method for this task, but we will
show an implementation using \lstinline!foldLeft!. The updater function
\emph{prepends} an element to a previous list:
\begin{lstlisting}
def reverse[A](xs: List[A]): List[A] =
  xs.foldLeft(Nil: List[A])((prev, x) => x :: prev)

scala> reverse(List(1, 2, 3))
res0: List[Int] = List(3, 2, 1) 
\end{lstlisting}
Without the explicit type annotation \lstinline!Nil:List[A]!, the
Scala compiler will decide that \lstinline!Nil! has type \lstinline!List[Nothing]!,
and the types will not match later in the code. In Scala, one often
finds that the initial value for \lstinline!.foldLeft! needs an explicit
type annotation.

The \lstinline!reverse! function can be used to obtain a tail-recursive
implementation of \lstinline!map! for \lstinline!List!. The idea
is to first use \lstinline!foldLeft! to accumulate transformed elements:
\begin{lstlisting}
scala> Seq(1, 2, 3).foldLeft(Nil:List[Int])((prev, x) => x*x :: prev)
res0: List[Int] = List(9, 4, 1)
\end{lstlisting}
The result is a reversed \lstinline!.map(x => x*x)!, so we need to
apply \lstinline!.reverse!:
\begin{lstlisting}
def map[A, B](xs: List[A])(f: A => B): List[B] =
  xs.foldLeft(Nil: List[B])((prev, x) => f(x) :: prev).reverse

scala> map(List(1, 2, 3))(x => x*x)
res2: List[Int] = List(1, 4, 9)
\end{lstlisting}
This achieves stack safety at the cost of traversing the list twice.
(This implementation is shown only as an example. The Scala library
implements \lstinline!.map! for \lstinline!List! using low-level
tricks in order to achieve better performance.)

\subsubsection{Example \label{subsec:Disjunctive-Example-non-empty-list-foldLeft}\ref{subsec:Disjunctive-Example-non-empty-list-foldLeft}\index{solved examples}}

A definition of the \textbf{non-empty list\index{non-empty list}}
is similar to \lstinline!List! except that the empty-list case is
replaced by a $1$-element case:
\begin{lstlisting}
sealed trait NEL[A]
final case class Last[A](head: A)               extends NEL[A]
final case class More[A](head: A, tail: NEL[A]) extends NEL[A]
\end{lstlisting}
Values of a non-empty list look like this:
\begin{lstlisting}
scala> val xs: NEL[Int] = More(1, More(2, Last(3)))  // [1, 2, 3]
xs: NEL[Int] = More(1,More(2,Last(3)))

scala> val ys: NEL[String] = Last("abc") // One element, ["abc"].
ys: NEL[String] = Last(abc)
\end{lstlisting}
To create non-empty lists more easily, we implement a conversion function
\lstinline!toNEL! from an ordinary list. To guarantee that a non-empty
list can be created, we give \lstinline!toNEL! \emph{two} arguments:
\begin{lstlisting}
def toNEL[A](x: A, rest: List[A]): NEL[A] = rest match {
  case Nil         => Last(x)
  case y :: tail   => More(x, toNEL(y, tail))
}   // Not tail-recursive: `toNEL()` is used inside `More(...)`.
\end{lstlisting}
To test:
\begin{lstlisting}
scala> toNEL(1, List()) // Result = [1].
res0: NEL[Int] = Last(1)

scala> toNEL(1, List(2, 3)) // Result = [1, 2, 3].
res1: NEL[Int] = More(1,More(2,Last(3)))
\end{lstlisting}

The \lstinline!head! method is safe for non-empty lists, unlike \lstinline!.head!
for an ordinary \lstinline!List!:
\begin{lstlisting}
def head[A]: NEL[A] => A = {
    case Last(x)        => x
    case More(x, _)     => x
}
\end{lstlisting}

We can also implement a tail-recursive \lstinline!foldLeft! function
for non-empty lists:
\begin{lstlisting}
@tailrec def foldLeft[A, R](n: NEL[A])(init: R)(f: (R, A) => R): R = n match {
    case Last(x)        => f(init, x)
    case More(x, tail)  => foldLeft(tail)(f(init, x))(f)
  }

scala> foldLeft(More(1, More(2, Last(3))))(0)(_ + _)
res2: Int = 6
\end{lstlisting}


\subsubsection{Example \label{subsec:Disjunctive-Example-non-empty-list}\ref{subsec:Disjunctive-Example-non-empty-list}}

Use \lstinline!foldLeft! to implement a \lstinline!reverse! function
for the type \lstinline!NEL!. The required type signature and a sample
test:
\begin{lstlisting}
def reverse[A]: NEL[A] => NEL[A] = ???

scala> reverse(toNEL(10, List(20, 30))) // Result must be [30, 20, 10].
res3: NEL[Int] = More(30,More(20,Last(10)))
\end{lstlisting}


\subparagraph{Solution}

We will use \lstinline!foldLeft! to build up the reversed list as
the accumulator value. It remains to choose the initial value of the
accumulator and the updater function. We have already seen the code
for reversing the ordinary list via the \lstinline!.foldLeft! method
(Section~\ref{subsec:Tail-recursion-with-list}),
\begin{lstlisting}
def reverse[A](xs: List[A]): List[A] = xs.foldLeft(Nil: List[A])((prev,x) => x :: prev)
\end{lstlisting}
However, we cannot reuse the same code for non-empty lists by writing
\lstinline!More(x, prev)! instead of \lstinline!x :: prev!, because
the \lstinline!foldLeft! operation works with non-empty lists differently.
Since lists are always non-empty, the updater function is always applied
to an initial value, and the code works incorrectly:
\begin{lstlisting}
def reverse[A](xs: NEL[A]): NEL[A] =
  foldLeft(xs)(Last(head(xs)):NEL[A])((prev,x) => More(x, prev))

scala> reverse(toNEL(10,List(20,30))) // Result = [30, 20, 10, 10].
res4: NEL[Int] = More(30,More(20,More(10,Last(10))))
\end{lstlisting}
The last element, \lstinline!10!, should not have been repeated.
It was repeated because the initial accumulator value already contained
the head element \lstinline!10! of the original list. However, we
cannot set the initial accumulator value to an empty list, since a
value of type \lstinline!NEL[A]! must be non-empty. It seems that
we need to handle the case of a one-element list separately. So we
begin by matching on the argument of \lstinline!reverse!, and apply
\lstinline!foldLeft! only when the list is longer than $1$ element:
\begin{lstlisting}
def reverse[A]: NEL[A] => NEL[A] = {
    case Last(x)         => Last(x)    // `reverse` is trivial.
    case More(x, tail)   =>            // Use foldLeft on `tail`.
      foldLeft(tail)(Last(x):NEL[A])((prev,x) => More(x, prev))
  }

scala> reverse(toNEL(10, List(20, 30))) // Result = [30, 20, 10].
res5: NEL[Int] = More(30,More(20,Last(10)))
\end{lstlisting}


\subsubsection{Exercise \label{subsec:Disjunctive-Exercise-non-empty-list-1}\ref{subsec:Disjunctive-Exercise-non-empty-list-1}\index{exercises}}

Implement a function \lstinline!toList! that converts a non-empty
list into an ordinary Scala \lstinline!List!. The required type signature
and a sample test:
\begin{lstlisting}
def toList[A](nel: NEL[A]): List[A] = ???

scala> toList(More(1, More(2, Last(3)))) // This is [1, 2, 3].
res6: List[Int] = List(1, 2, 3)
\end{lstlisting}
\begin{comment}
Solution:

\begin{lstlisting}
def toList[A](nel: NEL[A]): List[A] = nel match {
  case Last(x)         => List(x)
  case More(x, tail)   => x :: toList(tail)
} // Not tail-recursive.
def toList[A](nel: NEL[A]): List[A] = foldLeft(nel)(Nil:List[A]) {
  (prev, x) =>  x :: prev
}.reverse // Tail-recursive, but performs two traversals.
\end{lstlisting}
\end{comment}


\subsubsection{Exercise \label{subsec:Disjunctive-Exercise-map-for-NEL}\ref{subsec:Disjunctive-Exercise-map-for-NEL}}

Implement a \lstinline!map! function for the type \lstinline!NEL!.
Type signature and a sample test:
\begin{lstlisting}
def map[A,B](xs: NEL[A])(f: A => B): NEL[B] = ???

scala> map[Int, Int](toNEL(10, List(20, 30)))(_ + 5) // Result = [15, 25, 35].
res7: NEL[Int] = More(15,More(25,Last(35)))
\end{lstlisting}


\subsubsection{Exercise \label{subsec:Disjunctive-Exercise-non-empty-list-2}\ref{subsec:Disjunctive-Exercise-non-empty-list-2}}

Implement a function \lstinline!concat! that concatenates two non-empty
lists:
\begin{lstlisting}
def concat[A](xs: NEL[A], ys: NEL[A]): NEL[A] = ???

scala> concat(More(1, More(2, Last(3))), More(4, Last(5))) // Result is [1, 2, 3, 4, 5].
res8: NEL[Int] = More(1,More(2,More(3,More(4,Last(5)))))
\end{lstlisting}


\subsection{Binary trees}

We will consider four kinds of trees defined as recursive disjunctive
types: binary trees, rose trees, regular-shaped trees, and abstract
syntax trees.

Examples of a \textbf{binary tree\index{binary tree}} with leaves
of type \lstinline!A! can be drawn as {\tiny{}}{\tiny{} \Tree[ [ [ $a_1$ ] [ $a_2$ ] ]  [ $a_3$ ] ] }
or as {\small{}}{\tiny{} \Tree[ [ [ $a_1$ ] [ [ $a_2$ ] [ $a_3$ ] ] ] [ [ $a_4$ ] [ $a_5$ ] ] ] },
where $a_{i}$ are some values of type \lstinline!A!. 

An inductive definition says that a binary tree is either a leaf with
a value of type \lstinline!A! or a branch containing \emph{two} previously
defined binary trees. Translating this definition into code, we get
\begin{lstlisting}
sealed trait Tree2[A]
final case class Leaf[A](a: A)                       extends Tree2[A]
final case class Branch[A](x: Tree2[A], y: Tree2[A]) extends Tree2[A]
\end{lstlisting}
With this definition, the tree {\tiny{}}{\tiny{} \Tree[ [ [ $a_1$ ] [ $a_2$ ] ]  [ $a_3$ ] ] }
is created by the code expression
\begin{lstlisting}
Branch(Branch(Leaf("a1"), Leaf("a2")), Leaf("a3"))
\end{lstlisting}
while the expression
\begin{lstlisting}
Branch(Branch(Leaf("a1"),Branch(Leaf("a2"), Leaf("a3"))), Branch(Leaf("a4"), Leaf("a5")))
\end{lstlisting}
creates the tree {\tiny{} \Tree[ [ [ $a_1$ ] [ [ $a_2$ ] [ $a_3$ ] ] ] [ [ $a_4$ ] [ $a_5$ ] ] ] }.

Recursive functions on trees are translated into concise code. For
instance, the function \lstinline!foldLeft! for trees of type \lstinline!Tree2!
is implemented as
\begin{lstlisting}
def foldLeft[A, R](t: Tree2[A])(init: R)(f: (R, A) => R): R = t match {
  case Leaf(a)          => f(init, a)
  case Branch(t1, t2)   =>
    val r1 = foldLeft(t1)(init)(f) // Fold the left branch.
    foldLeft(t2)(r1)(f) // Starting from `r1`, fold the right branch.
}
\end{lstlisting}
Note that this function \emph{cannot} be made tail-recursive using
the accumulator trick, because \lstinline!foldLeft! needs to call
itself twice in the \lstinline!Branch! case. To test:
\begin{lstlisting}
val t: Tree2[String] = Branch(Branch(Leaf("a1"), Leaf("a2")), Leaf("a3"))

scala> foldLeft(t)("")(_ + " " + _)
res0: String = " a1 a2 a3"
\end{lstlisting}


\subsection{Rose trees}

A \textbf{rose tree}\index{rose tree} is similar to the binary tree
except the branches contain a non-empty list of trees. Because of
that, a rose tree can fork into arbitrarily many branches at each
node, rather than always into two branches as the binary tree does;
for example, {\tiny{} \Tree[ [ [ $a_1$ ] [ $a_2$ ] [ $a_3$ ] ] [ [ $a_4$ ] [ $a_5$ ] ] ] }
and {\tiny{} \Tree[ [ $a_1$ ] [ $a_2$ ] [ $a_3$ ] [ $a_4$ ] ] } are
rose trees.

A possible definition of a data type for the rose tree is
\begin{lstlisting}
sealed trait TreeN[A]
final case class Leaf[A](a: A)                extends TreeN[A]
final case class Branch[A](ts: NEL[TreeN[A]]) extends TreeN[A]
\end{lstlisting}


\subsubsection{Exercise \label{subsec:Disjunctive-Exercise-foldLeft-rose-tree}\ref{subsec:Disjunctive-Exercise-foldLeft-rose-tree}\index{exercises}}

Define the function \lstinline!foldLeft! for a rose tree, using \lstinline!foldLeft!
for the type \lstinline!NEL!. Type signature and a test:
\begin{lstlisting}
def foldLeft[A, R](t: TreeN[A])(init: R)(f: (R, A) => R): R = ???

scala> foldLeft(Branch(More(Leaf(1), More(Leaf(2), Last(Leaf(3))))))(0)(_ + _)
res0: Int = 6
\end{lstlisting}


\subsection{Regular-shaped trees}

Binary trees and rose trees may choose to branch or not to branch
at any given node, resulting in structures that may have different
branching depths at different nodes, such as {\tiny{} \Tree[ [ [ $a_1$ ] [ [ $a_2$ ] [ $a_3$ ] ] ] [ $a_4$ ] ] }.
A \textbf{regular-shaped tree}\index{regular-shaped tree} always
branches in the same way at every node until a chosen total depth,
e.g.~{\tiny{} \Tree[ [ [ $a_1$ ] [ $a_2$ ] ] [ [ $a_3$ ] [ $a_4$ ] ] ] },
where all nodes at depth $0$ and $1$ always branch into two, while
nodes at depth $2$ do not branch. The branching number is fixed for
a given type of a regular-shaped tree; in this example, the branching
number is $2$, so it is a regular-shaped \emph{binary} tree.

How can we define a data type representing a regular-shaped binary
tree? We need a tree that is either a single value, or a pair of values,
or a pair of pairs, etc. Begin with the non-recursive (but, of course,
impractical) definition
\begin{lstlisting}
sealed trait RTree[A]
final case class Leaf[A](x: A)                   extends RTree[A]
final case class Branch1[A](xs: (A, A))          extends RTree[A]
final case class Branch2[A](xs: ((A, A),(A, A))) extends RTree[A]
???                        // Need an infinitely long definition.
\end{lstlisting}
The case \lstinline!Branch1! describes a regular-shaped tree with
total depth $1$, the case \lstinline!Branch2! has total depth $2$,
and so on. Now, we cannot rewrite this definition as a recursive type
because the case classes do not have the same structure. The non-trivial
trick is to notice that each \lstinline!Branch!$_{n}$ case class
uses the previous case class's data structure \emph{with the type
parameter} set to \lstinline!(A, A)! instead of \lstinline!A!. So
we can rewrite this definition as
\begin{lstlisting}
sealed trait RTree[A]
final case class Leaf[A](x: A)                   extends RTree[A]
final case class Branch1[A](xs: Leaf[(A, A)])    extends RTree[A]
final case class Branch2[A](xs: Branch1[(A, A)]) extends RTree[A]
???                        // Need an infinitely long definition.
\end{lstlisting}
We can now apply the type recursion trick: replace the type \lstinline!Branch!$_{n-1}$\lstinline![(A, A)]!
in the definition of \lstinline!Branch!$_{n}$ by the type \lstinline!RTree[(A, A)]!
. This gives the type definition for a regular-shaped binary tree:
\begin{lstlisting}
sealed trait RTree[A]
final case class Leaf[A](x: A)                extends RTree[A]
final case class Branch[A](xs: RTree[(A, A)]) extends RTree[A]
\end{lstlisting}

Since we used some tricks to figure out the definition of \lstinline!RTree[A]!,
let us verify that this definition actually describes the recursive
disjunctive type we wanted. The only way to create a structure of
type \lstinline!RTree[A]! is to create a \lstinline!Leaf[A]! or
a \lstinline!Branch[A]!. A value of type \lstinline!Leaf[A]! is
a correct regularly-shaped tree. It remains to consider the case of
\lstinline!Branch[A]!. Creating a \lstinline!Branch[A]! requires
a previously created \lstinline!RTree! with values of type \lstinline!(A, A)!
instead of \lstinline!A!. By the inductive assumption, the previously
created \lstinline!RTree[A]! would have the correct shape. Now, it
is clear that if we replace the type parameter \lstinline!A! by the
pair \lstinline!(A, A)!, a regular-shaped tree such as {\tiny{} \Tree[ [ [ $a_1$ ] [ $a_2$ ] ] [ [ $a_3$ ] [ $a_4$ ] ] ] }
remains regular-shaped but becomes one level deeper, which can be
drawn (replacing each $a_{i}$ by a pair $a_{i}^{'},a_{i}^{"}$) as{\tiny{} \Tree[ [ [ [ $a_1^{'}$ ] [ $a_1^{''}$ ] ] [ [ $a_2^{'}$ ] [ $a_2^{''}$ ] ] ] [ [ [ $a_3^{'}$ ] [ $a_3^{''}$ ] ] [ [ $a_4^{'}$ ] [ $a_4^{''}$ ] ]  ] ] }.
We see that \lstinline!RTree[A]! is the correct definition of a regular-shaped
binary tree. 

\subsubsection{Example \label{subsec:Disjunctive-Example-map-regular-tree}\ref{subsec:Disjunctive-Example-map-regular-tree}\index{solved examples}}

Define a (non-tail-recursive) \lstinline!map! function for a regular-shaped
binary tree. The required type signature and a test:
\begin{lstlisting}
def map[A, B](t: RTree[A])(f: A => B): RTree[B] = ???

scala> map(Branch(Branch(Leaf(((1,2),(3,4))))))(_ * 10)
res0: RTree[Int] = Branch(Branch(Leaf(((10,20),(30,40)))))
\end{lstlisting}


\subparagraph{Solution}

Begin by pattern-matching on the tree:
\begin{lstlisting}
def map[A, B](t: RTree[A])(f: A => B): RTree[B] = t match {
  case Leaf(x)      => ???
  case Branch(xs)   => ???
}
\end{lstlisting}
In the base case, we have no choice but to return \lstinline!Leaf(f(x))!.
\begin{lstlisting}
def map[A, B](t: RTree[A])(f: A => B): RTree[B] = t match {
  case Leaf(x)      => Leaf(f(x))
  case Branch(xs)   => ???
}
\end{lstlisting}
In the inductive step, we are given a previous tree value \lstinline!xs:RTree[(A, A)]!.
It is clear that we need to apply \lstinline!map! recursively to
\lstinline!xs!. Let us try:

\begin{wrapfigure}{l}{0.61\columnwidth}%
\vspace{-0.5\baselineskip}

\begin{lstlisting}
def map[A, B](t: RTree[A])(f: A => B): RTree[B] = t match {
  case Leaf(x)      => Leaf(f(x))
  case Branch(xs)   => Branch(map(xs)(f))    // Type error!
}
\end{lstlisting}
\vspace{-0.75\baselineskip}
\end{wrapfigure}%

\noindent Here, \lstinline!map(xs)(f)! has an incorrect type of the
function \lstinline!f!. Since \lstinline!xs! has type \lstinline!RTree[(A, A)]!,
the recursive call \lstinline!map(xs)(f)! requires \lstinline!f!
to be of type \lstinline!((A, A)) => (B, B)! instead of \lstinline!A => B!.
So, we need to provide a function of the correct type instead of \lstinline!f!.
A function of type \lstinline!((A, A)) => (B, B)! will be obtained
out of \lstinline!f: A => B! if we apply \lstinline!f! to each part
of the tuple \lstinline!(A, A)!; the code for that function is \lstinline!{ case (x, y) => (f(x), f(y)) }!.
Therefore, we can implement \lstinline!map! as
\begin{lstlisting}
def map[A, B](t: RTree[A])(f: A => B): RTree[B] = t match {
  case Leaf(x)      => Leaf(f(x))
  case Branch(xs)   => Branch(map(xs){ case (x, y) => (f(x), f(y)) })
}
\end{lstlisting}


\subsubsection{Exercise \label{subsec:Disjunctive-Exercise-foldLeft-regular-tree-depth}\ref{subsec:Disjunctive-Exercise-foldLeft-regular-tree-depth}\index{exercises}}

Using tail recursion, compute the depth of a regular-shaped binary
tree of type \lstinline!RTree!. (An \lstinline!RTree! of depth $n$
has $2^{n}$ leaf values.) The required type signature and a test:
\begin{lstlisting}
@tailrec def depth[A](t: RTree[A]): Int = ???

scala> depth(Branch(Branch(Leaf((("a","b"),("c","d"))))))
res2: Int = 2
\end{lstlisting}
\begin{comment}
Solution:

\begin{lstlisting}
@tailrec def depth[A](t: RTree[A], acc: Int = 0): Int = t match {
  case Leaf(x)    => acc
  case Branch(xs) => depth(xs, acc + 1)
}
\end{lstlisting}
\end{comment}


\subsubsection{Exercise \label{subsec:Disjunctive-Exercise-foldLeft-regular-tree}\ref{subsec:Disjunctive-Exercise-foldLeft-regular-tree}{*}}

Define a tail-recursive function \lstinline!foldLeft! for a regular-shaped
binary tree. The required type signature and a test:
\begin{lstlisting}
@tailrec def foldLeft[A, R](t: RTree[A])(init: R)(f: (R, A) => R): R = ???

scala> foldLeft(Branch(Branch(Leaf(((1,2),(3,4))))))(0)(_ + _)
res0: Int = 10

scala> foldLeft(Branch(Branch(Leaf((("a","b"),("c","d"))))))("")(_ + _)
res1: String = abcd
\end{lstlisting}
\begin{comment}
Solution:

\begin{lstlisting}
@tailrec def foldLeft[A, R](t: RTree[A])(init: R)(f: (R, A) => R): R = t match {
  case Leaf(x)    => f(init, x)
  case Branch(xs) => foldLeft(xs)(init) { case (r, (a, b)) => f(f(r, a), b) } 
}
\end{lstlisting}
\end{comment}


\subsection{Abstract syntax trees}

Expressions in formal languages are represented by abstract syntax
trees. An \textbf{abstract syntax tree}\index{abstract syntax tree}\textbf{
}(or \textbf{AST} for short) is defined as either a leaf of one of
the available leaf types, or a branch of one of the available branch
types. All the available leaf and branch types must be specified as
part of the definition of an AST. In other words, one must specify
the data carried by leaves and branches, as well as the branching
numbers.

To illustrate how ASTs are used, let us rewrite Example~\ref{subsec:disj-Example-resultA}
via an AST. We view Example~\ref{subsec:disj-Example-resultA} as
a small sub-language that deals with ``safe integers'' and supports
the ``safe arithmetic'' operations \lstinline!Sqrt!, \lstinline!Add!,
\lstinline!Mul!, and \lstinline!Div!. Example calculations in this
sub-language are $\sqrt{16}*(1+2)=12$; $20+1/0=\text{error}$; and
$10+\sqrt{-1}=\text{error}$. 

We can implement this sub-language in two stages. The first stage
will create a data structure (an AST) that represents an unevaluated
expression\index{unevaluated expression} in the sub-language. The
second stage will evaluate that AST to obtain either a number or an
error message.

A straightforward way of defining a data structure for an AST is to
use a disjunctive type whose cases describe all the possible operations
of the sub-language. We will need one case class for each of \lstinline!Sqrt!,
\lstinline!Add!, \lstinline!Mul!, and \lstinline!Div!. An additional
operation, \lstinline!Num!, will lift ordinary integers into ``safe
integers''. So, we define the disjunctive type for ``arithmetic
sub-language expressions'' as
\begin{lstlisting}
sealed trait Arith
final case class Num(x: Int)             extends Arith
final case class Sqrt(x: Arith)          extends Arith
final case class Add(x: Arith, y: Arith) extends Arith
final case class Mul(x: Arith, y: Arith) extends Arith
final case class Div(x: Arith, y: Arith) extends Arith
\end{lstlisting}
A value of type \lstinline!Arith! is either a \lstinline!Num(x)!
for some integer \lstinline!x!, or an \lstinline!Add(x, y)! where
\lstinline!x! and \lstinline!y! are previously defined \lstinline!Arith!
expressions, or another operation.

This type definition is similar to the binary tree type
\begin{lstlisting}
sealed trait Tree
final case class Leaf(x: Int)             extends Tree
final case class Branch(x: Tree, y: Tree) extends Tree
\end{lstlisting}
if we rename \lstinline!Leaf! to \lstinline!Num! and \lstinline!Branch!
to \lstinline!Add!. However, the \lstinline!Arith! type contains
four different types of ``branches'', some with branching number
$1$ and others with branching number $2$. 

This example illustrates the structure of an AST: it is a tree of
a general shape, where leaves and branches are chosen from a specified
set of allowed possibilities. In this example, we have a single allowed
type of leaf (\lstinline!Num!) and four allowed types of branches
(\lstinline!Sqrt!, \lstinline!Add!, \lstinline!Mul!, and \lstinline!Div!).

This completes the first stage of implementing the sub-language of
``safe arithmetic''. We may now use the disjunctive type \lstinline!Arith!
to create expressions in the sub-language. For example, $\sqrt{16}*(1+2)$
is represented by
\begin{lstlisting}
scala> val x: Arith = Mul(Sqrt(Num(16)), Add(Num(1), Num(2)))
x: Arith = Mul(Sqrt(Num(16)),Add(Num(1),Num(2))) 
\end{lstlisting}
We can visualize \lstinline!x! as the abstract syntax tree{\tiny{} \Tree[.\texttt{Mul} [.\texttt{Sqrt} [.\texttt{Num} $16$ ] ] [.\texttt{Add} [ [.\texttt{Num} [ $1$ ] ] [.\texttt{Num} [ $2$ ] ] ] ] ] }. 

The expressions $20+1/0$ and $10*\sqrt{-1}$ are represented by
\begin{lstlisting}
scala> val y: Arith = Add(Num(20), Div(Num(1), Num(0)))
y: Arith = Add(Num(20),Div(Num(1),Num(0)))

scala> val z: Arith = Add(Num(10), Sqrt(Num(-1)))
z: Arith = Add(Num(10),Sqrt(Num(-1)))
\end{lstlisting}
As we see, the expressions \lstinline!x!, \lstinline!y!, and \lstinline!z!
\emph{remain} \emph{unevaluated}; each of them is a data structure
that encodes a tree of operations of the sub-language. These operations
will be evaluated at the second stage of implementing the sub-language.

To evaluate expressions in the ``safe arithmetic'', we can implement
a function with type signature \lstinline!run: Arith => Either[String, Int]!.
That function plays the role of an \textbf{interpreter}\index{interpreter}
or ``\textbf{runner}\index{runner}'' for programs written in the
sub-language. The runner will destructure the expression tree and
execute all the operations, taking care of possible errors. 

To implement \lstinline!run!, we need to define required arithmetic
operations on the type \lstinline!Either[String, Int]!. For instance,
we need to be able to add or multiply values of that type. Instead
of custom code from Example~\ref{subsec:disj-Example-resultA}, we
can use the standard \lstinline!.map! and \lstinline!.flatMap! methods
defined on \lstinline!Either!. For example, addition and multiplication
of two ``safe integers'' is implemented as 
\begin{lstlisting}
def add(x: Either[String, Int], y: Either[String, Int]):
    Either[String, Int] = x.flatMap { r1 => y.map(r2 => r1 + r2) }
def mul(x: Either[String, Int], y: Either[String, Int]):
    Either[String, Int] = x.flatMap { r1 => y.map(r2 => r1 * r2) }
\end{lstlisting}
while the ``safe division'' is
\begin{lstlisting}
def div(x: Either[String, Int], y: Either[String, Int]):
    Either[String, Int] = x.flatMap { r1 => y.flatMap(r2 =>
  if (r2 == 0) Left(s"error: $r1 / $r2") else Right(r1 / r2) )
}
\end{lstlisting}
With this code, we can implement the runner as
\begin{lstlisting}
def run: Arith => Either[String, Int] = {
  case Num(x)     => Right(x)
  case Sqrt(x)    => run(x).flatMap { r =>
    if (r < 0) Left(s"error: sqrt($r)") else Right(math.sqrt(r).toInt)
  }
  case Add(x, y)  => add(run(x), run(y))
  case Mul(x, y)  => mul(run(x), run(y))
  case Div(x, y)  => div(run(x), run(y))
}
\end{lstlisting}
Test it with the values \lstinline!x!, \lstinline!y!, \lstinline!z!
defined previously:
\begin{lstlisting}
scala> run(x)
res0: Either[String, Int] = Right(12)

scala> run(y)
res1: Either[String, Int] = Left("error: 1 / 0")

scala> run(z)
res2: Either[String, Int] = Left("error: sqrt(-1)")
\end{lstlisting}


\section{Summary}

What problems can we solve now?
\begin{itemize}
\item Represent values from a disjoint domain by a custom-defined disjunctive
type.
\item Use disjunctive types instead of exceptions to indicate failures.
\item Use standard disjunctive types \lstinline!Option!, \lstinline!Try!,
\lstinline!Either! and methods defined for them.
\item Define recursive disjunctive types (e.g.~lists and trees) and implement
recursive functions that work with them.
\end{itemize}
The following examples and exercises illustrate these tasks.

\subsection{Solved examples\index{solved examples}}

\subsubsection{Example \label{subsec:Example-disjunctive-1}\ref{subsec:Example-disjunctive-1}}

Define a disjunctive type \lstinline!DayOfWeek! representing the
seven days.

\subparagraph{Solution}

Since there is no information other than the label on each day, we
use empty case classes:
\begin{lstlisting}
sealed trait DayOfWeek
final case class Sunday()    extends DayOfWeek
final case class Monday()    extends DayOfWeek
final case class Tuesday()   extends DayOfWeek
final case class Wednesday() extends DayOfWeek
final case class Thursday()  extends DayOfWeek
final case class Friday()    extends DayOfWeek
final case class Saturday()  extends DayOfWeek
\end{lstlisting}
This data type is analogous to an enumeration type\index{enumeration type}
in C or C++:
\begin{lstlisting}[language={C++}]
typedef enum { Sunday, Monday, Tuesday, Wednesday, Thursday, Friday, Saturday } DayOfWeek;
\end{lstlisting}


\subsubsection{Example \label{subsec:Example-disjunctive-2}\ref{subsec:Example-disjunctive-2}}

Modify \lstinline!DayOfWeek! so that the values additionally represent
a restaurant name and total amount for Fridays and a wake-up time
on Saturdays. 

\subparagraph{Solution}

For the days where additional information is given, we use non-empty
case classes:
\begin{lstlisting}
sealed trait DayOfWeekX
final case class Sunday()                                extends DayOfWeekX
final case class Monday()                                extends DayOfWeekX
final case class Tuesday()                               extends DayOfWeekX
final case class Wednesday()                             extends DayOfWeekX
final case class Thursday()                              extends DayOfWeekX
final case class Friday(restaurant: String, amount: Int) extends DayOfWeekX
final case class Saturday(wakeUpAt: java.time.LocalTime) extends DayOfWeekX
\end{lstlisting}
This data type is no longer equivalent to an enumeration type.

\subsubsection{Example \label{subsec:disj-Example-rootsofq-2}\ref{subsec:disj-Example-rootsofq-2}}

Define a disjunctive type that describes the real roots of the equation
$ax^{2}+bx+c=0$, where $a$, $b$, $c$ are arbitrary real numbers.

\subparagraph{Solution}

Begin by solving the equation and enumerating all the possible cases.
It may happen that $a=b=c=0$, and then all $x$ are roots. If $a=b=0$
but $c\neq0$, the equation is $c=0$, which has no roots. If $a=0$
but $b\neq0$, the equation becomes $bx+c=0$, having a single root.
If $a\neq0$ and $b^{2}>4ac$, we have two distinct real roots. If
$a\neq0$ and $b^{2}=4ac$, we have one real root. If $b^{2}<4ac$,
we have no real roots. The resulting type definition can be written
as
\begin{lstlisting}
sealed trait RootsOfQ2
final case class AllRoots()                      extends RootsOfQ2
final case class ConstNoRoots()                  extends RootsOfQ2
final case class Linear(x: Double)               extends RootsOfQ2
final case class NoRealRoots()                   extends RootsOfQ2
final case class OneRootQ(x: Double)             extends RootsOfQ2
final case class TwoRootsQ(x: Double, y: Double) extends RootsOfQ2
\end{lstlisting}
This disjunctive type contains six parts, among which three parts
are empty tuples and two parts are single-element tuples; but this
is not a useless redundancy. We would lose information if we reused
\lstinline!Linear! for the two cases $a=0$, $b\neq0$ and $a\neq0$,
$b^{2}=4ac$, or if we reused \lstinline!NoRoots()! for representing
all three different no-roots cases.

\subsubsection{Example \label{subsec:Example-disjunctive-3}\ref{subsec:Example-disjunctive-3}}

Define a function \lstinline!rootAverage! that computes the average
value of all real roots of a general quadratic equation, where the
set of roots is represented by the type \lstinline!RootsOfQ2! defined
in Example~\ref{subsec:disj-Example-rootsofq-2}. The required type
signature is
\begin{lstlisting}
val rootAverage: RootsOfQ2 => Option[Double] = ???
\end{lstlisting}
The function should return \lstinline!None! if the average is undefined.

\subparagraph{Solution}

The average is defined only in cases \lstinline!Linear!, \lstinline!OneRootQ!,
and \lstinline!TwoRootsQ!. In all other cases, we must return \lstinline!None!.
We implement this via pattern matching:
\begin{lstlisting}
val rootAverage: RootsOfQ2 => Option[Double] = roots => roots match {
    case Linear(x)       => Some(x)
    case OneRootQ(x)     => Some(x)
    case TwoRootsQ(x, y) => Some((x + y) * 0.5)
    case _               => None
  }
\end{lstlisting}
We do not need to enumerate all other cases since the underscore ($\_$)
matches everything that the previous cases did not match.

The often-used code pattern of the form \lstinline!x => x match { case ... }!
can be shortened to the nameless function syntax \lstinline!{ case ...}!.
The code then becomes
\begin{lstlisting}
val rootAverage: RootsOfQ2 => Option[Double] = {
  case Linear(x)       => Some(x)
  case OneRootQ(x)     => Some(x)
  case TwoRootsQ(x, y) => Some((x + y) * 0.5)
  case _               => None
}
\end{lstlisting}
Test it:
\begin{lstlisting}
scala> Seq(NoRealRoots(),OneRootQ(1.0), TwoRootsQ(1.0, 2.0), AllRoots()).map(rootAverage)
res0: Seq[Option[Double]] = List(None, Some(1.0), Some(1.5), None)
\end{lstlisting}


\subsubsection{Example \label{subsec:Example-disjunctive-4}\ref{subsec:Example-disjunctive-4}}

Generate $100$ quadratic equations $x^{2}+bx+c=0$ with random coefficients
$b$, $c$ (uniformly distributed between $-1$ and $1$) and compute
the mean of the largest real roots from all these equations.

\subparagraph{Solution}

Use the type \lstinline!QEqu! and the \lstinline!solve! function
from Example~\ref{subsec:disj-Example-rootsofq-1}. A sequence of
equations with random coefficients is created by applying the method
\lstinline!Seq.fill!:
\begin{lstlisting}
def random(): Double = scala.util.Random.nextDouble() * 2 - 1
val coeffs: Seq[QEqu] = Seq.fill(100)(QEqu(random(), random()))
\end{lstlisting}
Now we can use the \lstinline!solve! function to compute all roots:
\begin{lstlisting}
val solutions: Seq[RootsOfQ] = coeffs.map(solve)
\end{lstlisting}
For each set of roots, compute the largest root:
\begin{lstlisting}
scala> val largest: Seq[Option[Double]] = solutions.map {
  case OneRoot(x)     => Some(x)
  case TwoRoots(x, y) => Some(math.max(x, y))
  case _              => None
}
largest: Seq[Option[Double]] = List(None, Some(0.9346072365885472), Some(1.1356234869160806), Some(0.9453181931646322), Some(1.1595052441078866), None, Some(0.5762252742788)...
\end{lstlisting}
It remains to remove the \lstinline!None! values and to compute the
mean of the resulting sequence. The Scala library defines the \lstinline!.flatten!
method that removes \lstinline!None!s and transforms \lstinline!Seq[Option[A]]!
into \lstinline!Seq[A]!:
\begin{lstlisting}
scala> largest.flatten
res0: Seq[Double] = List(0.9346072365885472, 1.1356234869160806, 0.9453181931646322, 1.1595052441078866, 0.5762252742788...
\end{lstlisting}
Now compute the mean of the last sequence. Since the \lstinline!.flatten!
operation is preceded by \lstinline!.map!, we can replace it by a
\lstinline!.flatMap!. The final code is
\begin{lstlisting}
val largest = Seq.fill(100)(QEqu(random(), random()))
  .map(solve)
  .flatMap {
    case OneRoot(x)     => Some(x)
    case TwoRoots(x, y) => Some(math.max(x, y))
    case _              => None
  }

scala> largest.sum / largest.size
res1: Double = 0.7682649774589514
\end{lstlisting}


\subsubsection{Example \label{subsec:Example-disjunctive-5}\ref{subsec:Example-disjunctive-5}}

Implement a function with type signature
\begin{lstlisting}
def f1[A, B]: Option[Either[A, B]] => Either[A, Option[B]] = ???
\end{lstlisting}
The function should preserve information as much as possible.

\subparagraph{Solution}

Begin by pattern matching on the argument:

\begin{wrapfigure}{l}{0.625\columnwidth}%
\vspace{-0.95\baselineskip}
\begin{lstlisting}[numbers=left,numberstyle={\small}]
def f1[A, B]: Option[Either[A, B]] => Either[A, Option[B]] = {
  case None                    => ???
  case Some(eab:Either[A, B])  => ???
}
\end{lstlisting}

\vspace{-1.15\baselineskip}
\end{wrapfigure}%

\noindent In line 3, we wrote the \textbf{type annotation}\index{type annotation}
\lstinline!:Either[A, B]! only for clarity; it is not required here
because the Scala compiler can deduce the type of the pattern variable
\lstinline!eab! from the fact that we are matching a value of type
\lstinline!Option[Either[A, B]]!.

In the scope of line 2, we need to return a value of type \lstinline!Either[A, Option[B]]!.
A value of that type must be either a \lstinline!Left(x)! for some
\lstinline!x:A!, or a \lstinline!Right(y)! for some \lstinline!y:Option[B]!,
where \lstinline!y! must be either \lstinline!None! or \lstinline!Some(z)!
with a \lstinline!z:B!. However, in our case the code is of the form
\lstinline!case None => ???!, and we cannot produce any values \lstinline!x:A!
or \lstinline!z:B! since \lstinline!A! and \lstinline!B! are arbitrary,
unknown types. The only remaining possibility is to return \lstinline!Right(y)!
with \lstinline!y = None!, and so the code must be
\begin{lstlisting}
  ...
  case None => Right(None) // No other choice here.
\end{lstlisting}

In the next scope, we can perform pattern matching on the value \lstinline!eab!:

\begin{wrapfigure}{l}{0.46\columnwidth}%
\vspace{-0.8\baselineskip}
\begin{lstlisting}
  ...
  case Some(eab: Either[A, B]) = eab match {
    case Left(a)   => ???
    case Right(b)  => ???
  }
\end{lstlisting}

\vspace{-0.9\baselineskip}
\end{wrapfigure}%

\noindent It remains to figure out what expressions to write in each
case. In the case \lstinline!Left(a) => ???!, we have a value of
type \lstinline!A!, and we need to compute a value of type \lstinline!Either[A, Option[B]]!.
We execute the same argument as before: The return value must be \lstinline!Left(x)!
for some \lstinline!x:A!, or \lstinline!Right(y)! for some \lstinline!y:Option[B]!.
At this point, we have a value of type \lstinline!A! but no values
of type \lstinline!B!. So we have two possibilities: to return \lstinline!Left(a)!
or to return \lstinline!Right(None)!. If we decide to return \lstinline!Left(a)!,
the code is

\begin{wrapfigure}{l}{0.655\columnwidth}%
\vspace{-0.6\baselineskip}
\begin{lstlisting}[numbers=left,numberstyle={\small}]
def f1[A, B]: Option[Either[A, B]] => Either[A, Option[B]] = {
  case None       => Right(None) // No other choice here.
  case Some(eab)  => eab match {
    case Left(a)  => Left(a)   // Could return Right(None) here.
    case Right(b) => ???
  }
}
\end{lstlisting}

\vspace{-0.8\baselineskip}
\end{wrapfigure}%

\noindent Let us decide whether to return \lstinline!Left(a)! or
\lstinline!Right(None)! in line 4. Both choices will satisfy the
required return type \lstinline!Either[A, Option[B]]!. However, if
we return \lstinline!Right(None)! in that line, we will ignore the
given value \lstinline!a:A!, i.e.~we will lose information.\index{information loss}
So we return \lstinline!Left(a)! in line 4.

Reasoning similarly for line 5, we find that we may return \lstinline!Right(None)!
or \lstinline!Right(Some(b))!. The first choice ignores the given
value of \lstinline!b:B!. To preserve information, we make the second
choice:
\begin{lstlisting}[numbers=left,numberstyle={\small}]
def f1[A, B]: Option[Either[A, B]] => Either[A, Option[B]] = {
  case None       => Right(None)
  case Some(eab)  => eab match {
      case Left(a)    => Left(a)
      case Right(b)   => Right(Some(b))
  }
}
\end{lstlisting}

We can now refactor this code into a somewhat more readable form by
using nested patterns: 
\begin{lstlisting}
def f1[A, B]: Option[Either[A, B]] => Either[A, Option[B]] = {
  case None             => Right(None)
  case Some(Left(a))    => Left(a)
  case Some(Right(b))   => Right(Some(b))
}
\end{lstlisting}


\subsubsection{Example \label{subsec:Example-disjunctive-6}\ref{subsec:Example-disjunctive-6}}

Implement a function with the type signature 
\begin{lstlisting}
def f2[A, B]: (Option[A], Option[B]) => Option[(A, B)] = ???
\end{lstlisting}
The function should preserve information as much as possible.

\subparagraph{Solution}

Begin by pattern matching on the argument:

\begin{wrapfigure}{l}{0.6\columnwidth}%
\vspace{-0.6\baselineskip}
\begin{lstlisting}[numbers=left,numberstyle={\small}]
def f2[A, B]: (Option[A], Option[B]) => Option[(A, B)] = {
  case (Some(a), Some(b)) => ???
  ...
\end{lstlisting}

\vspace{-0.8\baselineskip}
\end{wrapfigure}%

\noindent In line 2, we have values \lstinline!a:A! and \lstinline!b:B!,
and we need to compute a value of type \lstinline!Option[(A, B)]!.
A value of that type is either \lstinline!None! or \lstinline!Some((x, y))!
where we would need to obtain values \lstinline!x:A! and \lstinline!y:B!.
Since \lstinline!A! and \lstinline!B! are arbitrary types, we cannot
produce new values \lstinline!x! and \lstinline!y! from scratch.
The only way of obtaining \lstinline!x:A! and \lstinline!y:B! is
to set \lstinline!x = a! and \lstinline!y = b!. So, our choices
are to return \lstinline!Some((a, b))! or \lstinline!None!. We reject
returning \lstinline!None! since that would unnecessarily lose information\index{information loss}.
Thus, we continue writing code as
\begin{lstlisting}[numbers=left,numberstyle={\small}]
def f2[A, B]: (Option[A], Option[B]) => Option[(A, B)] = {
  case (Some(a), Some(b)) => Some((a, b))
  case (Some(a), None)    => ???
  ...
\end{lstlisting}
In line 3, we have a value \lstinline!a:A! but no values of type
\lstinline!B!. Since the type \lstinline!B! is arbitrary, we cannot
produce any values of type \lstinline!B! to return a value of the
form \lstinline!Some((x, y))!. So, \lstinline!None! is the only
computable value of type \lstinline!Option[(A, B)]! in line 3. We
continue to write the code:
\begin{lstlisting}[numbers=left,numberstyle={\small}]
def f2[A, B]: (Option[A], Option[B]) => Option[(A, B)] = {
  case (Some(a), Some(b)) => Some((a, b))
  case (Some(a), None)    => None // No other choice here.
  case (None, Some(b))    => ???
  case (None, None)       => ???
}
\end{lstlisting}
In lines 4\textendash 5, we find that there is no choice other than
returning \lstinline!None!. So we can simplify the code:
\begin{lstlisting}
def f2[A, B]: (Option[A], Option[B]) => Option[(A, B)] = {
  case (Some(a), Some(b)) => Some((a, b))
  case _                  => None // No other choice here.
}
\end{lstlisting}


\subsection{Exercises\index{exercises}}

\subsubsection{Exercise \label{subsec:Exercise-disjunctive-1}\ref{subsec:Exercise-disjunctive-1}}

Define a disjunctive type \lstinline!CellState! representing the
visual state of one cell in the \emph{Minesweeper}\footnote{\texttt{\href{https://en.wikipedia.org/wiki/Minesweeper_(video_game)}{https://en.wikipedia.org/wiki/Minesweeper\_(video\_game)}}}
game: A cell can be closed (showing nothing), or show a bomb, or be
open and show the number of bombs in neighbor cells.

\subsubsection{Exercise \label{subsec:Exercise-disjunctive-2}\ref{subsec:Exercise-disjunctive-2}}

Define a function from \lstinline!Seq[Seq[CellState]]! to \lstinline!Int!,
counting the total number of cells with zero neighbor bombs shown.

\subsubsection{Exercise \label{subsec:Exercise-disjunctive-3}\ref{subsec:Exercise-disjunctive-3}}

Define a disjunctive type \lstinline!RootOfLinear! representing all
possibilities for the solution of the equation $ax+b=0$ for arbitrary
real $a$, $b$. (The possibilities are: no roots; one root; all $x$
are roots.) Implement the solution as a function \lstinline!solve1!
with type signature 
\begin{lstlisting}
def solve1: ((Double, Double)) => RootOfLinear = ???
\end{lstlisting}


\subsubsection{Exercise \label{subsec:Exercise-disjunctive-4}\ref{subsec:Exercise-disjunctive-4}}

Given a \lstinline!Seq[(Double, Double)]! containing pairs $\left(a,b\right)$
of the coefficients of $ax+b=0$, produce a \lstinline!Seq[Double]!
containing the roots of that equation when a unique root exists. Use
the type \lstinline!RootOfLinear! and the function \lstinline!solve1!
defined in Exercise~\ref{subsec:Exercise-disjunctive-3}.

\subsubsection{Exercise \label{subsec:Exercise-disjunctive-4-1-1}\ref{subsec:Exercise-disjunctive-4-1-1}}

The case class \lstinline!Subscriber! was defined in Example~\ref{subsec:Disjunctive-Example-option-1}.
Given a \lstinline!Seq[Subscriber]!, compute the sequence of email
addresses for all subscribers that did \emph{not} provide a phone
number.

\subsubsection{Exercise \label{subsec:Exercise-disjunctive-Procedure-DSL}\ref{subsec:Exercise-disjunctive-Procedure-DSL}{*}}

In this exercise, a ``procedure\index{procedure}'' is a function
of type \lstinline!Unit => Unit!; an example of a procedure is \lstinline!{ () => println("hello") }!.
Define a disjunctive type \lstinline!Proc! for an abstract syntax
tree representing three operations on procedures: 1) \lstinline!Func[A](f)!,
create a procedure from a function \lstinline!f! of type \lstinline!Unit => A!,
where \lstinline!A! is a type parameter. (Note that the type \lstinline!Proc!
does not have any type parameters.) 2) \lstinline!Sequ(p1, p2)!,
execute two procedures sequentially. 3) \lstinline!Para(p1, p2)!,
execute two procedures in parallel. Then implement a ``runner''
that converts a \lstinline!Proc! into a \lstinline!Future[Unit]!,
running the computations either sequentially or in parallel as appropriate.
Test with this code:
\begin{lstlisting}
sealed trait Proc; final case class ??? // etc.
def runner: Proc => Future[Unit] = ???
val proc1: Proc = Func{_ => Thread.sleep(200); println("hello1")}
val proc2: Proc = Func{_ => Thread.sleep(400); println("hello2")}

scala> runner(Sequ(Para(proc2, proc1), proc2))
hello1
hello2
hello2
\end{lstlisting}


\subsubsection{Exercise \label{subsec:Exercise-disjunctive-5}\ref{subsec:Exercise-disjunctive-5}}

Implement functions that have a given type signature, preserving information
as much as possible: 
\begin{lstlisting}
def f1[A, B]: Option[(A, B)] => (Option[A], Option[B]) = ???
def f2[A, B]: Either[A, B] => (Option[A], Option[B]) = ???
def f3[A,B,C]: Either[A, Either[B,C]] => Either[Either[A,B], C] = ???
\end{lstlisting}


\subsubsection{Exercise \label{subsec:Exercise-disjunctive-EvenList}\ref{subsec:Exercise-disjunctive-EvenList}}

Define a parameterized type \lstinline!EvenList[A]! representing
a list of values of type \lstinline!A! that is guaranteed to have
an even number of elements (zero, two, four, etc.). Implement functions
\lstinline!foldLeft! and \lstinline!map! for \lstinline!EvenList!.%
\begin{comment}
Solution:
\begin{lstlisting}
sealed trait EvenList[A]
final case class Lempty[A]() extends EvenList[A]
final case class Lpair[A](x: A, y: A, tail: EvenList[A]) extends EvenList

def fmap[A, B](f: A => B): EvenList[A] => EvenList[B] = {
case Lempty() => Lempty[B]()
case Lpair(x, y, tail) => Lpair[B](f(x), f(y), fmap(f)(tail))
}
\end{lstlisting}
\end{comment}


\section{Discussion}

\subsection{Disjunctive types as mathematical sets}

To understand the properties of disjunctive types from the mathematical
point of view, consider a function whose argument is a disjunctive
type, such as
\begin{lstlisting}
def isDoubleRoot(r: RootsOfQ) = ...
\end{lstlisting}
The type of the argument \lstinline!r:RootsOfQ! represents the mathematical
domain of the function, that is, the set of admissible values of the
argument \lstinline!r!. What kind of domain is that? The set of real
roots of a quadratic equation $x^{2}+bx+c=0$ can be empty, or it
can contain a single real number $x$, or a pair of real numbers $\left(x,y\right)$.
Geometrically, a number $x$ is pictured as a point on a line (a one-dimensional
space), and pair of numbers $\left(x,y\right)$ is pictured as a point
on a Cartesian plane (a two-dimensional space). The no-roots case
corresponds to a zero-dimensional space, which is pictured as a single
point (see Figure~\ref{fig:RootsOfQ-disjoint-domain}). The point,
the line, and the plane do not intersect (have no common points);
together, they form the domain of the possible roots. Such domains
are called \textbf{\index{disjoint domain}disjoint}. So, the set
of real roots of a quadratic equation $x^{2}+bx+c=0$ is a disjoint
domain containing three parts.

\begin{figure}
\begin{centering}
   \newrgbcolor{lightpastel}{0.90 0.93 0.87}
   \psset{unit=0.5\textwidth}
   \begin{pspicture}(0,0)(1,1)
      \pscircle(0.3,0.5){0.005}
      \psline{->}(0.5,0.2)(0.5,0.8)
      \rput(0.535,0.77){$x$}
      \pspolygon[fillstyle=solid,fillcolor=lightpastel](0.8,0.8)(0.8,0.3)(0.7,0.2)(0.7,0.7)(0.8,0.8)
      \psline{->}(0.7,0.2)(0.775,0.275)
      \rput(0.67,0.28){$x$} \rput(0.79,0.24){$y$}
      \psline{->}(0.7,0.2)(0.7,0.3)
      \rput(0.30,0.45){\smaller\texttt{NoRoots()}}
      \rput(0.49,0.85){\smaller\texttt{OneRoot(x)}}
      \rput(0.75,0.15){\smaller\texttt{TwoRoots(x, y)}}
   \end{pspicture}
\par\end{centering}
\caption{The disjoint domain represented by the \lstinline!RootsOfQ! type.\label{fig:RootsOfQ-disjoint-domain}}
\end{figure}

In the mathematical notation, a one-dimensional real space is denoted
by $\mathbb{R}$, a two-dimensional space by $\mathbb{R}^{2}$, and
a zero-dimensional space by $\mathbb{R}^{0}$. At first, we may think
that the mathematical representation of the type \lstinline!RootsOfQ!
is a union of the three sets, $\mathbb{R}^{0}\cup\mathbb{R}^{1}\cup\mathbb{R}^{2}$.
But an ordinary union of sets would not always work correctly because
we need to distinguish the parts of the union unambiguously, even
if some parts have the same type. For instance, the disjunctive type
shown in Example~\ref{subsec:disj-Example-rootsofq-2} cannot be
correctly represented by the mathematical union
\[
\mathbb{R}^{0}\cup\mathbb{R}^{0}\cup\mathbb{R}^{1}\cup\mathbb{R}^{0}\cup\mathbb{R}^{1}\cup\mathbb{R}^{2}
\]
because $\mathbb{R}^{0}\cup\mathbb{R}^{0}=\mathbb{R}^{0}$ and $\mathbb{R}^{1}\cup\mathbb{R}^{1}=\mathbb{R}^{1}$,
so 
\[
\mathbb{R}^{0}\cup\mathbb{R}^{0}\cup\mathbb{R}^{1}\cup\mathbb{R}^{0}\cup\mathbb{R}^{1}\cup\mathbb{R}^{2}=\mathbb{R}^{0}\cup\mathbb{R}^{1}\cup\mathbb{R}^{2}\quad.
\]
This representation has lost the distinction between e.g.~\lstinline!Linear(x)!
and \lstinline!OneRootQ(x)!.

In the Scala code, each part of a disjunctive type must be distinguished
by a unique name such as \lstinline!NoRoots!, \lstinline!OneRoot!,
and \lstinline!TwoRoots!. To represent this mathematically, we can
attach a distinct label to each part of the union. Labels are symbols
without any special meaning, and we can just assume that labels are
names of Scala case classes. Parts of the union are then represented
by sets of pairs such as $(\text{\texttt{OneRoot}},x)_{x\in\mathbb{R}^{1}}$.
Then the domain \lstinline!RootsOfQ! is expressed as
\[
\text{\texttt{RootsOfQ}}=(\text{\texttt{NoRoots}},u)_{u\in\mathbb{R}^{0}}\cup(\text{\texttt{OneRoot}},x)_{x\in\mathbb{R}^{1}}\cup(\text{\texttt{TwoRoots}},\left(x,y\right))_{\left(x,y\right)\in\mathbb{R}^{2}}\quad.
\]
This is an ordinary union of mathematical sets, but each of the sets
has a unique label, so no two values from different parts of the union
could possibly be equal. This kind of labeled union is called a \index{disjoint union}\textbf{disjoint
union}. Each element of the disjoint union is a pair of the form \lstinline!(label, data)!,
where the label uniquely identifies the part of the union, and the
data can have any chosen type such as $\mathbb{R}^{1}$. If we use
disjoint unions, we cannot confuse different parts of the union even
if their data have the same type, because labels are required to be
distinct.

Disjoint unions are not often used in mathematics, but they are needed
in software engineering because real-life data often belongs to disjoint
domains.

\paragraph{Named \texttt{Unit} types}

At first sight, it may seem strange that the zero-dimensional space
is represented by a set containing \emph{one} point. Why should we
not use an empty set (rather than a set with one point) to represent
the case where the equation has no real roots? The reason is that
we are required to represent not only the values of the roots but
also the information \emph{about} the existence of the roots. The
case with no real roots needs to be represented by some \emph{value}
of type \lstinline!RootsOfQ!. This value cannot be missing, which
would happen if we used an empty set to represent the no-roots case.
It is natural to use the named empty tuple \lstinline!NoRoots()!
to represent this case, just as we used a named $2$-tuple \lstinline!TwoRoots(x, y)!
to represent the case of two roots.

Consider the value $u$ used by the mathematical set $\left(\text{\texttt{NoRoots}},u\right)_{u\in\mathbb{R}^{0}}$.
Since $\mathbb{R}^{0}$ consists of a single point, there is only
\emph{one} possible value of $u$. Similarly, the \lstinline!Unit!
type in Scala has only one distinct value, written as \lstinline!()!.
A case class with no parts, such as \lstinline!NoRoots!, has only
one distinct value, written as \lstinline!NoRoots()!. This Scala
value is fully analogous to the mathematical notation $\left(\text{\texttt{NoRoots}},u\right)_{u\in\mathbb{R}^{0}}$.

We see that case classes with no parts are quite similar to \lstinline!Unit!
except for an added name. For this reason, they can be viewed as ``named
unit'' types.\index{unit type!named}

\subsection{Disjunctive types in other programming languages}

Disjunctive types and the associated pattern matching turns out to
be one of the defining features of functional programming languages.
Languages that were not designed for functional programming do not
support these features, while ML, OCaml, Haskell, F\#, Scala, Swift,
Elm, and PureScript support disjunctive types and pattern matching
as part of the language design. 

It is remarkable that the named tuple types (also called ``structs''
or ``records'') are provided in almost every programming language,
while disjunctive types are almost never present except in languages
designed for the FP paradigm. (Ada and Pascal are the only languages
that have disjunctive types without other FP features.\footnote{\texttt{\href{https://en.wikipedia.org/wiki/Comparison_of_programming_languages_(basic_instructions)\#Other_types}{https://en.wikipedia.org/wiki/Comparison\_of\_programming\_languages\_(basic\_instructions)\#Other\_types}}})

The \lstinline!union! types in C and C++ are not disjunctive types
because it is not possible to determine which part of the union is
represented by a given value. A \lstinline!union! declaration in
C looks like this,
\begin{lstlisting}
union { int x; double y; long z; } di;
\end{lstlisting}
The problem is that we cannot determine whether a given value \lstinline!di!
represents an \lstinline!int!, a \lstinline!double!, or a \lstinline!long!.
This leads to errors that are hard to detect.

Programming languages of the C family (C, C++, Objective C, Java)
support \textbf{enumeration} (\lstinline!enum!) types\index{enumeration type},
which are a limited form of disjunctive types. An \lstinline!enum!
type declaration in Java looks like this:
\begin{lstlisting}
enum Color { RED, GREEN, BLUE; } 
\end{lstlisting}
In Scala, this is equivalent to a disjunctive type containing three
\emph{empty} tuples,
\begin{lstlisting}
sealed trait Color
final case class RED()   extends Color
final case class GREEN() extends Color
final case class BLUE()  extends Color
\end{lstlisting}
If \lstinline!enum! types were ``enriched'' with extra data, so
that the tuples could be non-empty, we would obtain the full functionality
of disjunctive types. A definition of \lstinline!RootsOfQ! could
then look like this: 
\begin{lstlisting}
enum RootsOfQ {      // This is not valid in Java!
  NoRoots(), OneRoot(Double x), TwoRoots(Double x, Double y);
}
\end{lstlisting}
A future version of Scala 3 will have a shorter syntax for disjunctive
types\footnote{\texttt{\href{https://dotty.epfl.ch/docs/reference/enums/adts.html}{https://dotty.epfl.ch/docs/reference/enums/adts.html}}}
that indeed looks like an ``enriched \lstinline!enum!'':
\begin{lstlisting}
enum RootsOfQ {
  case NoRoots
  case OneRoot(x: Double)
  case TwoRoots(x: Double, y: Double)
}
\end{lstlisting}
For comparison, the syntax for a disjunctive type equivalent to \lstinline!RootsOfQ!
in OCaml and Haskell is
\begin{lstlisting}[language=Caml]
(* OCaml *)
type RootsOfQ = NoRoots | OneRoot of float | TwoRoots of float*float
\end{lstlisting}
\begin{lstlisting}[language=Haskell]
-- Haskell
data RootsOfQ = NoRoots | OneRoot Double | TwoRoots (Double, Double)
\end{lstlisting}
This is more concise than the Scala syntax. When reasoning about disjunctive
types, it is inconvenient to write out long type definitions. Chapter~\ref{chap:3-3-The-formal-logic-curry-howard}
will define a mathematical notation designed for efficient reasoning
about types.

\subsection{Disjunctions and conjunctions in formal logic\label{subsec:Disjunctions-and-conjunctions}}

In logic, a \textbf{proposition\index{proposition (in logic)}} is
a logical formula that could be true or false. A \textbf{disjunction\index{disjunction (in logic)}}
of propositions $A$, $B$, $C$ is denoted by $A\vee B\vee C$ and
is true if and only if \emph{at least one} of $A$, $B$, $C$ is
true. A \textbf{conjunction}\index{conjunction (in logic)} of $A$,
$B$, $C$ is denoted by $A\wedge B\wedge C$ and is true if and only
if \emph{all} of the propositions $A$, $B$, $C$ are true.

There is a similarity between a disjunctive data type and a logical
disjunction of propositions. A value of the disjunctive data type
\lstinline!RootsOfQ! can be constructed only if we have one of the
values \lstinline!NoRoots()!, \lstinline!OneRoot(x)!, or \lstinline!TwoRoots(x, y)!.
Let us now rewrite the previous sentence as a logical formula. Denote
by ${\cal CH}(A)$ the logical proposition ``this ${\cal C}$ode
${\cal H}$as a value of type \lstinline!A!'', where ``this code''
refers to a particular function in a program. So, the proposition
``the function \emph{can} return a value of type \lstinline!RootsOfQ!''
is denoted by ${\cal CH}(\text{\texttt{RootsOfQ}})$. We can then
write the above sentence about \lstinline!RootsOfQ! as the logical
formula
\begin{equation}
{\cal CH}(\text{\texttt{RootsOfQ}})={\cal CH}(\text{\texttt{NoRoots}})\vee{\cal CH}(\text{\texttt{OneRoot}})\vee{\cal CH}(\text{\texttt{TwoRoots}})\quad.\label{eq:curry-howard-example-disjunction}
\end{equation}

There is also a a similarity between logical \emph{conjunctions} and
tuple types. Consider the named tuple (i.e.~a case class) \lstinline!TwoRoots(x: Double, y: Double)!.
When can we have a value of type \lstinline!TwoRoots!? Only if we
have two values of type \lstinline!Double!. Rewriting this sentence
as a logical formula, we get
\[
{\cal CH}(\text{\texttt{TwoRoots}})={\cal CH}(\text{\texttt{Double}})\wedge{\cal CH}(\text{\texttt{Double}})\quad.
\]
Formal logic admits the simplification
\[
{\cal CH}(\text{\texttt{Double}})\wedge{\cal CH}(\text{\texttt{Double}})={\cal CH}(\text{\texttt{Double}})\quad.
\]
However, no such simplification will be available in the general case,
e.g.
\begin{lstlisting}
case class Data3(x: Int, y: String, z: Double)
\end{lstlisting}
For this type, we will have the formula 
\begin{equation}
{\cal CH}(\text{\texttt{Data3}})={\cal CH}(\text{\texttt{Int}})\wedge{\cal CH}(\text{\texttt{String}})\wedge{\cal CH}(\text{\texttt{Double}})\quad.\label{eq:curry-howard-example-case-class}
\end{equation}

We find that tuples are related to logical conjunctions in the same
way as disjunctive types are related to logical disjunctions. This
is the main reason for choosing the name ``disjunctive types''.\footnote{These types are also called ``variants'', ``sum types'', ``co-product
types'', and ``tagged union types''.}

The correspondence between disjunctions, conjunctions, and data types
is explained in more detail in Chapter~\ref{chap:3-3-The-formal-logic-curry-howard}.
For now, we note that the operations of conjunction and disjunction
are not sufficient to produce all possible logical expressions. To
obtain a complete logic, it is also necessary to have a logical negation
$\neg A$ (``$A$ is not true'') or, equivalently, a logical implication
$A\rightarrow B$ (``if $A$ is true than $B$ is true''). It turns
out that the logical implication $A\rightarrow B$ is related to the
function type \lstinline!A => B!. In Chapter~\ref{chap:Higher-order-functions},
we will study function types in depth.
