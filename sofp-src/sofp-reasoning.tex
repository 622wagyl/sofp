
\chapter{Reasoning about code. Techniques of symbolic derivation\label{chap:Reasoning-about-code}}

\global\long\def\gunderline#1{\mathunderline{greenunder}{#1}}%
\global\long\def\bef{\forwardcompose}%
\global\long\def\bbnum#1{\custombb{#1}}%

In previous chapters, we have performed symbolic derivations of laws
for different functions. To make those derivations more manageable,
we gradually introduced special notations and techniques of reasoning.
This short chapter is a summary of these notations and techniques.

\section{Mathematical code notation}

\subsection{The eight code constructions}

The eight basic constructions\index{eight code constructions} introduced
in Section~\ref{subsec:The-rules-of-proof} serve as a foundation
for \textbf{purely functional} coding style. All major techniques
and design patterns of functional programming can be implemented using
only these constructions, i.e.~by purely functional\index{purely functional programs}
programs. We will now define the code notation (summarized in Table~\ref{tab:Mathematical-notation-for-basic-code-constructions})
for each of these eight constructions. 

\begin{table}
\begin{centering}
\begin{tabular}{|c|c|c|}
\hline 
\textbf{\small{}Constructions} & \textbf{\small{}Scala examples} & \textbf{\small{}Code notation}\tabularnewline
\hline 
\hline 
{\small{}Use a constant} & {\small{}}\lstinline!()!{\small{} or }\lstinline!true!{\small{}
or }\lstinline!"abc"!{\small{} or }\lstinline!123! & {\small{}$1$, $\text{true}$, $\text{"abc"}$, $123$}\tabularnewline
\hline 
{\small{}Use a given argument} & {\small{}}\lstinline!def f(x: A) = { ... x ... }! & {\small{}$f(x^{:A})\triangleq...~x~...$}\tabularnewline
\hline 
{\small{}Create a function} & {\small{}}\lstinline!(x: A) => expr(x)! & {\small{}$x^{:A}\rightarrow\text{expr}(x)$}\tabularnewline
\hline 
{\small{}Use a function} & {\small{}}\lstinline!f(x)!{\small{} or }\lstinline!x.pipe(f)!{\small{}
(Scala 2.13)} & {\small{}$f(x)$ or $x\triangleright f$}\tabularnewline
\hline 
{\small{}Create a tuple} & {\small{}}\lstinline!val p: (A, B) = (a, b)! & {\small{}$p^{:A\times B}\triangleq a\times b$}\tabularnewline
\hline 
{\small{}Use a tuple} & {\small{}}\lstinline!p._1!{\small{} or }\lstinline!p._2! & {\small{}$p\triangleright\nabla_{1}$ or $p\triangleright\nabla_{2}$}\tabularnewline
\hline 
{\small{}Create a disjunctive value} & {\small{}}\lstinline!Left[A, B](x)!{\small{} or }\lstinline!Right[A, B](y)! & {\small{}$x^{:A}+\bbnum 0^{:B}$ or $\bbnum 0^{:A}+y^{:B}$}\tabularnewline
\hline 
{\small{}Use a disjunctive value} & {\small{}\hspace*{-0.013\linewidth}}%
\begin{minipage}[c][1\totalheight][b]{0.33\columnwidth}%
{\small{}\vspace{0.14\baselineskip}
}
\begin{lstlisting}
val p: Either[A, B] = ... 
val q: C = p match {
    case Left(x)   => f(x)
    case Right(y)  => g(y)
}
\end{lstlisting}
{\small{}\vspace{-0.1\baselineskip}
}%
\end{minipage}{\small{} \hspace*{-0.009\linewidth}} & {\small{}$q^{:C}\triangleq p^{:A+B}\triangleright\begin{array}{|c||c|}
 & C\\
\hline A & x\rightarrow f(x)\\
B & y\rightarrow g(y)
\end{array}$}\tabularnewline
\hline 
\end{tabular}
\par\end{centering}
\caption{Mathematical notation for the eight basic code constructions.\label{tab:Mathematical-notation-for-basic-code-constructions}}
\end{table}


\paragraph{1) Use a constant}

At any place in code, we may use a fixed constant value of a primitive
type, such as \lstinline!Int!, \lstinline!String!, or \lstinline!Unit!.
We may also use a ``named unit\index{unit type!named}'', e.g.~\lstinline!None!
of type \lstinline!Option[A]! for any type \lstinline!A!. All named
unit values are denoted by $1$ and are viewed as having type $\bbnum 1$. 

With this construction, we can create \index{constant function}\textbf{constant
functions} (functions that ignore their argument):

\begin{wrapfigure}{l}{0.5\columnwidth}%
\vspace{-0.65\baselineskip}
\begin{lstlisting}
def c_1(x: String): Int = 123
\end{lstlisting}

\vspace{-0.25\baselineskip}
\end{wrapfigure}%

~\vspace{-0.35\baselineskip}
\[
c_{1}(x^{:\text{String}})\triangleq123\quad.
\]
\vspace{-0.85\baselineskip}


\paragraph{2) Use a given argument}

In any expression that has a bound variable (e.g.~body of a function
with a bound argument), we may use the bound variable at any place,
as many times as we need.

\begin{wrapfigure}{l}{0.5\columnwidth}%
\vspace{-0.65\baselineskip}
\begin{lstlisting}
def c_2[A](x: String, y: Int): Int = 123 + y + y
\end{lstlisting}

\vspace{-0.25\baselineskip}
\end{wrapfigure}%

~\vspace{-0.35\baselineskip}
\[
c_{2}(x^{:\text{String}},y^{:\text{Int}})\triangleq123+y+y\quad.
\]
\vspace{-0.85\baselineskip}


\paragraph{3) Create a function}

We can always make a nameless function \lstinline!{ x => expr }!
out of a variable, say \lstinline!x!, and any expression \lstinline!expr!
that may use \lstinline!x! as a free variable (i.e.~a variable that
is expected to be already defined outside the expression). E.g.~the
expression \lstinline!123 + y + y! uses \lstinline!y! as a free
variable because \lstinline!123 + y + y! only makes sense if \lstinline!y!
is already defined. So, we can create a nameless function

\begin{wrapfigure}{l}{0.5\columnwidth}%
\vspace{-0.65\baselineskip}
\begin{lstlisting}
{ y: Int => 123 + y + y }
\end{lstlisting}

\vspace{-0.25\baselineskip}
\end{wrapfigure}%

~\vspace{-0.35\baselineskip}
\[
y^{:\text{Int}}\rightarrow123+y+y\quad.
\]
\vspace{-0.85\baselineskip}

If the expression \lstinline!expr! already contains \lstinline!x!
as a bound variable, the function \lstinline!{ x => expr }! will
have a name clash, which can be avoided by renaming the bound variable.
As an example, consider an expression \lstinline!expr == { x => x }!
that already contains a nameless function with bound variable \lstinline!x!.
If we want to make a function out of that expression, we could write
\lstinline!x => { x => x }!, but it would be confusing. It is helpful
to rename the bound variables inside \lstinline!expr!, e.g.~\lstinline!expr == { z => z }!,
and in this way avoid the name clash:

\begin{wrapfigure}{l}{0.5\columnwidth}%
\vspace{-0.65\baselineskip}
\begin{lstlisting}
val f = { x: Int => { z: Int => z } }
\end{lstlisting}

\vspace{-0.25\baselineskip}
\end{wrapfigure}%

~\vspace{-0.35\baselineskip}
\[
f\triangleq x^{:\text{Int}}\rightarrow z^{:\text{Int}}\rightarrow z\quad.
\]
\vspace{-0.85\baselineskip}


\paragraph{4) Use a function}

If a function is already defined, we can use it by applying it to
an argument.

\begin{wrapfigure}{l}{0.5\columnwidth}%
\vspace{-0.65\baselineskip}
\begin{lstlisting}
val f = { y: Int => 123 + y + y }
f(100)  // Evaluates to 323.
\end{lstlisting}

\vspace{-0.25\baselineskip}
\end{wrapfigure}%

~\vspace{-0.65\baselineskip}
\[
f\triangleq y^{:\text{Int}}\rightarrow123+y+y\quad,\quad\quad f(100)=323\quad.
\]
\vspace{-0.85\baselineskip}


\paragraph{5) Create a tuple}

Given two values \lstinline!a: A! and \lstinline!b: B!, we can always
create the tuple \lstinline!(a, b)! (and the tuple \lstinline!(b, a)!
as well). In the code notation, those tuples are written as $a\times b$
and $b\times a$.

\paragraph{6) Use a tuple}

Given a tuple of, say, two values, \lstinline!p == (a, b)!, we can
extract each of the values. The Scala syntax \lstinline!p._1! and
\lstinline!p._2! corresponds to the code notation $p\triangleright\nabla_{1}$
and $p\triangleright\nabla_{2}$. The auxiliary functions $\nabla_{i}$
(where $i=1,2,...$) may be used for tuples of any size. Example code
defining these functions:

\begin{wrapfigure}{l}{0.5\columnwidth}%
\vspace{-0.85\baselineskip}
\begin{lstlisting}
def project_1[A, B]: ((A, B)) => A = {
    case (a, b) => a
} // Same as `_._1`
def project_2[A, B]: ((A, B)) => B = {
    case (a, b) => b
} // Same as `_._2`
\end{lstlisting}

\vspace{-0.5\baselineskip}
\end{wrapfigure}%

~\vspace{0.55\baselineskip}
\[
\nabla_{1}^{A,B}\triangleq a^{:A}\times b^{:B}\rightarrow a\quad,
\]
\[
\nabla_{2}^{A,B}\triangleq a^{:A}\times b^{:B}\rightarrow b\quad.
\]
\vspace{0.2\baselineskip}

The code notation $a\times b$ is used in an \emph{argument} of a
function to destructure a tuple.

\paragraph{7) Create a disjunctive value}

If a disjunctive type such as $A+B+C$ has been defined in Scala,
its named ``constructors'' (i.e.~case classes) are used to create
values of the type:

\begin{wrapfigure}{l}{0.5\columnwidth}%
\vspace{-0.75\baselineskip}
\begin{lstlisting}
sealed trait S
final case class P(w: Int, x: Int)  extends S
final case class Q(y: String)       extends S
final case class R(z: Int)          extends S

val s: S = P(10, 20)  // Create a value of S.
val t: S = R(30)      // One more value of S.
\end{lstlisting}

\vspace{-0\baselineskip}
\end{wrapfigure}%

~\vspace{0.35\baselineskip}
\[
S\triangleq\text{Int}\times\text{Int}+\text{String}+\text{Int}\quad,
\]
\begin{align*}
s^{:S} & \triangleq10\times20+\bbnum 0^{:\text{String}}+\bbnum 0^{:\text{Int}}\quad,\\
t^{:S} & \triangleq\bbnum 0^{:\text{Int}\times\text{Int}}+\bbnum 0^{:\text{String}}+30\quad.
\end{align*}
\vspace{-0.9\baselineskip}

The code notation for disjunctive values, e.g.~$\bbnum 0+\bbnum 0+x$,
is more verbose than the Scala syntax such as \lstinline!R(x)!. The
advantage is that we may explicitly annotate all types and show clearly
the part of the disjunction that we are creating. Another advantage
is that the notation $\bbnum 0+\bbnum 0+x$ is similar to a row vector,
$\,\begin{array}{|ccc|}
\bbnum 0 & \bbnum 0 & x\end{array}\:$, which is well adapted to the matrix notation for functions.

\paragraph{8) Use a disjunctive value}

Once created, disjunctive values can be used in a pattern matching
expression (Scala's \lstinline!match! / \lstinline!case!). Recall
that functions that take a disjunctive value as an argument (``\index{disjunctive functions}\textbf{disjunctive
functions}'') may be written \emph{without} the \lstinline!match!
keyword:

\begin{wrapfigure}{l}{0.5\columnwidth}%
\vspace{-0.65\baselineskip}
\begin{lstlisting}
val compute: Option[Int] => Option[Int] = {
    case None      => Some(100)
    case Some(x)   => Some(x / 2)
}
\end{lstlisting}

\vspace{-0.75\baselineskip}
\end{wrapfigure}%

~\vspace{-1.45\baselineskip}
\[
\text{compute}^{:\bbnum 1+\text{Int}\rightarrow\bbnum 1+\text{Int}}\triangleq\,\begin{array}{|c||cc|}
 & \bbnum 1 & \text{Int}\\
\hline \bbnum 1 & \bbnum 0 & 1\rightarrow100\\
\text{Int} & \bbnum 0 & x\rightarrow\frac{x}{2}
\end{array}\quad.
\]
\vspace{-0.8\baselineskip}

We will use this example to see how disjunctive functions are written
in the matrix notation\index{matrix notation}.

Each row of the matrix corresponds to a part of the disjunctive type
matched by one of the \lstinline!case! expressions. In this example,
the disjunctive type \lstinline!Option[Int]! has two parts, the named
unit \lstinline!None! (denoted by $\bbnum 1$) and the case class
\lstinline!Some[Int]!, which is equivalent to the type \lstinline!Int!.
So, the matrix has two rows labeled $\bbnum 1$ and $\text{Int}$,
showing that the function's argument type is $\bbnum 1+\text{Int}$.

The columns of the matrix correspond to the parts of the disjunctive
type \emph{returned} by the function. In this example, the return
type is also \lstinline!Option[Int]!, that is, $\bbnum 1+\text{Int}$,
so the matrix has two columns labeled $\bbnum 1$ and $\text{Int}$.
If the return type is not disjunctive, the matrix will have one column.

What are the matrix elements? The idea of the matrix notation is to
translate the \lstinline!case! expressions line by line from the
Scala code. Look at the first \lstinline!case! line as if it were
a standalone partial function,
\begin{lstlisting}
{ case None => Some(100) }
\end{lstlisting}
Since \lstinline!None! is a named unit, this function is written
in the code notation as $1\rightarrow\bbnum 0+100^{:\text{Int}}$. 

The second line is written in the form of a partial function as
\begin{lstlisting}
{ case Some(x) => Some(x / 2) }
\end{lstlisting}
The only pattern variable on the left side is \lstinline!x!, so we
can denote this function by $x^{:\text{Int}}\rightarrow\bbnum 0+(x/2)^{:\text{Int}}$. 

To obtain the matrix notation, we may simply write the two partial
functions in the two rows:

\begin{wrapfigure}{l}{0.5\columnwidth}%
\vspace{-0.85\baselineskip}
\begin{lstlisting}
val compute: Option[Int] => Option[Int] = {
    case None      => Some(100)
    case Some(x)   => Some(x / 2)
}
\end{lstlisting}

\vspace{-0.75\baselineskip}
\end{wrapfigure}%

~\vspace{-1.35\baselineskip}
\[
\text{compute}^{:\bbnum 1+\text{Int}\rightarrow\bbnum 1+\text{Int}}\triangleq\,\begin{array}{|c||c|}
 & \bbnum 1+\text{Int}\\
\hline \bbnum 1 & 1\rightarrow\bbnum 0+100\\
\text{Int} & x\rightarrow\bbnum 0+\frac{x}{2}
\end{array}\quad.
\]
\vspace{-0.9\baselineskip}

This is already a valid matrix notation for the function $f$. So
far, the matrix has two rows and one column. However, we notice that
each row's return value is \emph{known} to be in a specific part of
the disjunctive type $\bbnum 1+\text{Int}$ (in this example, both
rows happen to return values of type $\bbnum 0+\text{Int}$). So,
we can split the column into two and obtain a clearer and more useful
notation for this function:
\[
\text{compute}^{:\bbnum 1+\text{Int}\rightarrow\bbnum 1+\text{Int}}\triangleq\,\begin{array}{|c||cc|}
 & \bbnum 1 & \text{Int}\\
\hline \bbnum 1 & \bbnum 0 & 1\rightarrow100\\
\text{Int} & \bbnum 0 & x^{:\text{Int}}\rightarrow\frac{x}{2}
\end{array}\quad.
\]
The void type $\bbnum 0$ is written symbolically to indicate that
the disjunctive part of that column is not returned. In this way,
matrix notation can easily express functions that work with disjunctive
types.

Partial functions are written in the matrix notation by writing $\bbnum 0$
in the missing rows:

\begin{wrapfigure}{l}{0.5\columnwidth}%
\vspace{-0.65\baselineskip}
\begin{lstlisting}
def get[A]: Option[A] => A = {
    case Some(x) => x
}     // Partial function; does not work on None.
\end{lstlisting}

\vspace{-0.75\baselineskip}
\end{wrapfigure}%

~\vspace{-1.35\baselineskip}
\[
\text{get}^{:\bbnum 1+A\rightarrow A}\triangleq\,\begin{array}{|c||c|}
 & A\\
\hline \bbnum 1 & \bbnum 0\\
A & x^{:A}\rightarrow x
\end{array}\,=\,\begin{array}{|c||c|}
 & A\\
\hline \bbnum 1 & \bbnum 0\\
A & \text{id}
\end{array}\quad.
\]
\vspace{-0.9\baselineskip}

Scala's \lstinline!match! expression is equivalent to an application
of a disjunctive function:

\begin{wrapfigure}{l}{0.5\columnwidth}%
\vspace{-0.85\baselineskip}
\begin{lstlisting}
val p: Option[Int] = Some(64)
val q: Option[Int] = p match {
    case None      => Some(100)
    case Some(x)   => Some(x / 2)
}
\end{lstlisting}

\vspace{-2.75\baselineskip}
\end{wrapfigure}%

~\vspace{-0.25\baselineskip}
\[
p\triangleq\bbnum 0+64^{:\text{Int}}\quad,\quad q\triangleq p\triangleright\,\begin{array}{|c||cc|}
 & \bbnum 1 & \text{Int}\\
\hline \bbnum 1 & \bbnum 0 & 1\rightarrow100\\
\text{Int} & \bbnum 0 & x\rightarrow\frac{x}{2}
\end{array}\quad.
\]
\vspace{-0.1\baselineskip}
It is convenient to put the argument to the left of the partial function,
as in the Scala code.

The matrix notation allows us to compute such function applications
directly. We view the disjunctive value as a ``row vector'' and
follow the standard rules for the vector-matrix product:
\[
(\bbnum 0+64)\triangleright\,\begin{array}{||cc|}
\bbnum 0 & 1\rightarrow100\\
\bbnum 0 & x\rightarrow\frac{x}{2}
\end{array}\,=\,\begin{array}{|cc|}
\bbnum 0 & 64\end{array}\,\triangleright\,\begin{array}{||cc|}
\bbnum 0 & 1\rightarrow100\\
\bbnum 0 & x\rightarrow\frac{x}{2}
\end{array}\,=\,\begin{array}{|cc|}
\bbnum 0 & 64\triangleright(x\rightarrow\frac{x}{2})\end{array}\,=\,\begin{array}{|cc|}
\bbnum 0 & 32\end{array}\,=(\bbnum 0+32)\quad.
\]
The pipe ($\triangleright$) operation plays the role of the ``multiplication''
of matrix elements; and we drop any terms containing $\bbnum 0$.
We also omitted type annotations since we already checked that the
types match.\newpage{}

\subsection{Function composition and the pipe operator}

In addition to the basic code constructions, our derivations will
often need to work with function compositions and lifted functions.
It is often faster to perform calculations with functions when we
do not write all of their arguments explicitly; e.g.~writing the
right identity law as $f\bef\text{id}=\text{id}$ instead of $\forall x.\,\text{id}(f(x))=f(x)$.
This is known as ``calculating in \index{point-free calculations}\textbf{point-free}
style'' (meaning ``argument-free''). Many laws can be formulated
and used more easily in the point-free form. 

Calculations in point-free style almost always involve composing functions.
This book prefers to use the ``forward'' function composition $(f\bef g$)
defined for arbitrary $f^{:A\rightarrow B}$ and $g^{:B\rightarrow C}$
by

\begin{wrapfigure}{l}{0.5\columnwidth}%
\vspace{-0.65\baselineskip}
\begin{lstlisting}
f andThen g == { x => g(f(x)) }
\end{lstlisting}

\vspace{-0.25\baselineskip}
\end{wrapfigure}%

~\vspace{-0.35\baselineskip}
\[
f\bef g\triangleq x\rightarrow g(f(x))\quad.
\]
\vspace{-0.85\baselineskip}

An important tool for calculations is the pipe\index{pipe notation}
operation, $x\triangleright f$, which places the argument ($x$)
to the \emph{left} of a function ($f$). It is then natural to apply
further functions at \emph{right}, for example $(x\triangleright f)\triangleright g$
meaning $g(f(x))$. In Scala, methods such as \lstinline!.map! and
\lstinline!.filter! are often combined in this way:

\begin{wrapfigure}{l}{0.5\columnwidth}%
\vspace{-0.65\baselineskip}
\begin{lstlisting}
x.map(f).filter(p)
\end{lstlisting}

\vspace{-0.25\baselineskip}
\end{wrapfigure}%

~\vspace{-0.35\baselineskip}
\[
x\triangleright\text{fmap}\,(f)\triangleright\text{filter}\,(p)\quad.
\]
\vspace{-0.85\baselineskip}

To enable this common usage, the $\triangleright$ operation is defined
to group towards the left. So, the parentheses in $(x\triangleright f)\triangleright g$
are not needed, and we can write simply $x\triangleright f\triangleright g$. 

Since $x\triangleright f\triangleright g=g(f(x))$, we find that the
composition $f\bef g$ satisfies
\[
x\triangleright f\triangleright g=x\triangleright(f\bef g)\quad.
\]
This formula is needed quite often, so we will not write parentheses
in such formulas: $x\triangleright(f\bef g)=x\triangleright f\bef g$.
The pipe operation ($\triangleright$) groups weaker than the composition
operation ($\bef$).\index{pipe notation!operator precedence}

Another common simplification occurs with function compositions of
the form
\[
(x\rightarrow t\triangleright f)\bef g=x\rightarrow(t\triangleright f\triangleright g)=x\rightarrow t\triangleright f\bef g\quad.
\]

How can we verify this and other similar computations where the operations
$\triangleright$ and $\bef$ are combined in some way? Instead of
memorizing a large set of identities, we can rely on knowing only
one rule that says how arguments are symbolically substituted as parameters
into functions, for example:
\begin{align*}
{\color{greenunder}\text{substitute }x\text{ instead of }a:}\quad & \gunderline x\triangleright(\gunderline a\rightarrow f(\gunderline a))=f(x)\quad.\\
{\color{greenunder}\text{substitute }f(x)\text{ instead of }y:}\quad & (x\rightarrow\gunderline{f(x)})\bef(\gunderline y\rightarrow g(\gunderline y))=x\rightarrow g(f(x))\quad.
\end{align*}
Whenever there is a doubt (is $x\triangleright(f\triangleright g)$
or $(x\bef f)\triangleright g$ the correct formula..?), we can always
write functions in an expanded form, $x\rightarrow f(x)$ instead
of $f$, and perform calculations more verbosely. After getting experience
with the $\triangleright$ and $\bef$ operations, the reader will
start using them more freely without writing functions in expanded
form.

The matrix notation and the pipe operation are well adapted to \emph{forward}
function compositions. As an example, let us compute the composition
of the functions \lstinline!compute! and \lstinline!get[Int]! shown
above: 
\[
\text{compute}\bef\text{get}=\,\begin{array}{|c||cc|}
 & \bbnum 1 & \text{Int}\\
\hline \bbnum 1 & \bbnum 0 & 1\rightarrow100\\
\text{Int} & \bbnum 0 & x\rightarrow\frac{x}{2}
\end{array}\,\bef\,\begin{array}{|c||c|}
 & \text{Int}\\
\hline \bbnum 1 & \bbnum 0\\
\text{Int} & \text{id}
\end{array}\,=\,\begin{array}{|c||c|}
 & \text{Int}\\
\hline \bbnum 1 & (1\rightarrow100)\bef\text{id}\\
\text{Int} & (x\rightarrow\frac{x}{2})\bef\text{id}
\end{array}=\,\begin{array}{|c||c|}
 & \text{Int}\\
\hline \bbnum 1 & 1\rightarrow100\\
\text{Int} & x\rightarrow\frac{x}{2}
\end{array}\quad.
\]
In this computation, we used the composition ($\bef$) instead of
the ``multiplication'' of matrix elements.

Why does the rule for matrix multiplication work for function compositions?
The reason is the equivalence $x\triangleright f\triangleright g=x\triangleright f\bef g$.
We have defined the matrix form of functions to work with the ``row-vector''
form of disjunctive types. So, the matrix rule works for the computation
$x\triangleright f$ (where $x$ is a value of a disjunctive type).
The result of computing $x\triangleright f$ is again a row vector,
which we can pipe into another matrix $g$ as $x\triangleright f\triangleright g$.
The matrix multiplication is associative; so, the result of $x\triangleright f\triangleright g$
is the same as the result of piping $x$ into the matrix product of
$f$ and $g$. Therefore, the matrix product of $f$ and $g$ must
yield the function $f\bef g$.

A ``non-disjunctive'' function (i.e.~one not taking or returning
disjunctive types) may be written as a $1\times1$ matrix, so its
composition with disjunctive functions can be computed via the same
rules. 

\subsection{Functor and contrafunctor liftings}

Functions lifted to a functor (or a contrafunctor) and their compositions
are found so commonly in derivations that one needs better notation
than $x\triangleright\text{fmap}_{F}(f)$ or its Scala analog \lstinline!x.map(f)!.
The notation used in this book is $x\triangleright f^{\uparrow F}$
for functors $F$, and $f^{\downarrow C}$ for contrafunctors $C$.
This notation graphically emphasizes the function $f$ being lifted
and also mentions the name of the relevant functor or the contrafunctor.
Compositions of lifted functions are visually easy to recognize, for
instance:
\[
f^{\uparrow L}\bef g^{\uparrow L}=\left(f\bef g\right)^{\uparrow L}\quad,\quad\quad f^{\uparrow L}\bef g^{\uparrow L}\bef h^{\uparrow L}=\left(f\bef g\bef h\right)^{\uparrow L}\quad.
\]
In these formulas, the label $^{\uparrow L}$ clearly indicates the
possibility of pulling several functions under a single lifting. We
can also split a lifted composition into a composition of liftings. 

The lifting notation helps us recognize that these steps are possible
by looking at the formula. Of course, it still remains to find the
correct sequence of steps in a given derivation or proof.

\section{Derivation techniques}

\subsection{Auxiliary functions for handling products}

The functions denoted by $\nabla_{1}$, $\nabla_{2}$, $\Delta$,
and $\boxtimes$ proved to be useful in derivations that involve tuples.
(However, these functions are unlikely to be frequently used in practical
programming.) 

We already saw the definition and the implementation of the functions
$\nabla_{1}$ and $\nabla_{2}$. 

The ``diagonal'' function $\Delta$ is a right inverse for $\nabla_{1}$
and $\nabla_{2}$:

\begin{wrapfigure}{l}{0.5\columnwidth}%
\vspace{-0.65\baselineskip}
\begin{lstlisting}
def delta[A]: A => (A, A) = { x => (x, x) }
\end{lstlisting}

\vspace{-0.25\baselineskip}
\end{wrapfigure}%

~\vspace{-1.15\baselineskip}
\[
\Delta^{A}:A\rightarrow A\times A\quad,\quad\quad\Delta\triangleq a^{:A}\rightarrow a\times a\quad.
\]
\vspace{-1.15\baselineskip}

It is clear that extracting any part of a pair \lstinline!delta(x) == (x, x)!
will give back the original \lstinline!x!. This property can be written
as an equation or a ``law'',

\begin{wrapfigure}{l}{0.5\columnwidth}%
\vspace{-0.65\baselineskip}
\begin{lstlisting}
delta(x)._1 == x
\end{lstlisting}

\vspace{-0.25\baselineskip}
\end{wrapfigure}%

~\vspace{-0.35\baselineskip}
\[
\nabla_{1}(\Delta(x))=x\quad.
\]
\vspace{-0.85\baselineskip}

We can transform this law into a point-free equation by first using
the pipe notation,
\[
\nabla_{1}(\Delta(x))=\nabla_{1}(x\triangleright\Delta)=x\triangleright\Delta\triangleright\nabla_{1}=x\triangleright\Delta\bef\nabla_{1}\quad,
\]
and then rewriting the equation $x\triangleright\Delta\bef\nabla_{1}=x$
into the point-free form using the identity function, 
\begin{align*}
{\color{greenunder}\Delta\text{ is a right inverse of }\nabla_{1}:}\quad & \Delta\bef\nabla_{1}=\text{id}\quad.
\end{align*}
The same property holds for $\nabla_{2}$.

The \index{pair product of functions}\textbf{pair product} operation
$f\boxtimes g$ is defined for any functions $f^{:A\rightarrow P}$
and $f^{:B\rightarrow Q}$ by
\begin{lstlisting}
def pair_product[A,B,P,Q](f: A => P, g: B => Q): ((A, B)) => (P, Q) = {
    case (a, b) => (f(a), g(b))
}
\end{lstlisting}
\[
f\boxtimes g:A\times B\rightarrow P\times Q\quad,\quad\quad f\boxtimes g\triangleq a\times b\rightarrow f(a)\times g(b)\quad.
\]
Two properties of this operation follow directly from its definition:
\begin{align*}
{\color{greenunder}\text{composition law}:}\quad & (f^{:A\rightarrow P}\boxtimes g^{:B\rightarrow Q})\bef(m^{:P\rightarrow X}\boxtimes n^{:Q\rightarrow Y})=(f\bef m)\boxtimes(g\bef n)\quad.\\
{\color{greenunder}\text{identity law}:}\quad & \text{id}^{A}\boxtimes\text{id}^{B}=\text{id}^{A\times B}\quad.
\end{align*}
An equivalent way of defining $f\boxtimes g$ is via this Scala code,
\begin{lstlisting}
def pair_product[A,B,P,Q](f: A => P, g: B => Q)(p: (A, B)): (P, Q)  =  (f(p._1), g(p._2))
\end{lstlisting}
\[
f\boxtimes g=p^{:A\times B}\rightarrow f(p\triangleright\nabla_{1})\times g(p\triangleright\nabla_{2})=p\rightarrow(p\triangleright\nabla_{1}\triangleright f)\times(p\triangleright\nabla_{2}\triangleright g)\quad.
\]


\subsection{Deriving laws for functions with known implementations}

The task is to prove a given law (an equation) for a function whose
code is known. An example of such a derivation is the ``naturality
law'' of $\Delta$, which states that for any function $f^{:A\rightarrow B}$
we have
\begin{equation}
f\bef\Delta=\Delta\bef(f\boxtimes f)\quad.\label{eq:naturality-law-of-Delta}
\end{equation}

We will prefer to derive laws in the code notation rather than in
Scala syntax. The code notation covers all purely functional code,
i.e.~all programs that use only the eight basic code constructions.

Laws for fully parametric functions are usually written without type
annotations. However, it is important to check that types match. So
we begin by finding suitable type parameters for Eq.~(\ref{eq:naturality-law-of-Delta}).

Since it is given that $f$ has type $A\rightarrow B$, the function
$\Delta$ in the left-hand side of Eq.~(\ref{eq:naturality-law-of-Delta})
must take arguments of type $B$ and thus returns a value of type
$B\times B$. We see that the left-hand side must be a function of
type $A\rightarrow B\times B$. So, the $\Delta$ in the right-hand
side must take arguments of type $A$. It then returns a value of
type $A\times A$, which is consumed by $f\boxtimes f$. In this way,
we see that all types match. We can show the resulting types as a
type diagram, and write the law with type annotations:

\begin{wrapfigure}{L}{0.3\columnwidth}%
\vspace{-2.2\baselineskip}
\[
\xymatrix{\xyScaleY{1.6pc}\xyScaleX{4.0pc}A\ar[d]\sb(0.5){f}\ar[r]\sp(0.45){\Delta} & A\times A\ar[d]\sb(0.5){f\boxtimes f}\\
B\ar[r]\sp(0.45){\Delta} & B\times B
}
\]
\vspace{-0.5\baselineskip}
\end{wrapfigure}%

~\vspace{-0.3\baselineskip}
\[
f^{:A\rightarrow B}\bef\Delta^{:B\rightarrow B\times B}=\Delta^{:A\rightarrow A\times A}\bef(f\boxtimes f)\quad.
\]

\noindent To prove the law, we need to use the known code of the function
$\Delta$. We substitute that code into the left-hand side of the
law and into the right-hand side of the law, hoping to transform these
two expressions until they are the same.

We will now perform this computation in the Scala syntax and in the
code notation.

\begin{wrapfigure}{L}{0.6\columnwidth}%
\vspace{-0.6\baselineskip}
\begin{lstlisting}
x.pipe(f andThen delta)
  == (f(x)).pipe { a => (a, a) }
  == (f(x), f(x)) // Left-hand side.
x.pipe(delta andThen { case (a, b) => (f(a), f(b)) })
  == (x, x).pipe { case (a, b) => (f(a), f(b)) }
  == (f(x), f(x)) // Right-hand side.
\end{lstlisting}
\vspace{-3\baselineskip}
\end{wrapfigure}%

~\vspace{-1.4\baselineskip}
\begin{align*}
 & x\triangleright f\bef\Delta=f(x)\,\gunderline{\triangleright\,(b}\rightarrow b\times b)\\
 & \quad=f(x)\times f(x)\quad.\\
 & \gunderline{x\triangleright\Delta}\bef(f\boxtimes f)\\
 & \quad=(x\times x)\gunderline{\,\triangleright\,(a\times b}\rightarrow f(a)\times f(b))\\
 & \quad=f(x)\times f(x)\quad.
\end{align*}
\vspace{-1.5\baselineskip}

At each step of the derivation, there is only one symbolic transformation
we can perform; this is typical in proofs of laws. In the example
above, each step either substituted a definition of a known function
or applied some function to its argument and computed the result.
The green underline is a hint indicating a sub-expression that will
change at the next step.

\subsection{Working with disjunctive functions in matrix notation\label{subsec:Working-with-disjunctive-functions}}

The matrix notation is a general way of working with disjunctive types
in point-free style. However, the notation is unusual and needs some
getting used to. At first, it may help to write out all types in matrices.
This helps translate between Scala code and the matrix notation.

In many cases, the rules of matrix multiplication and function composition
are sufficient for calculating with disjunctive functions. For example,
consider the following functions \lstinline!swap[A]! and \lstinline!merge[A]!,

\begin{wrapfigure}{L}{0.47\columnwidth}%
\vspace{-0.85\baselineskip}
\begin{lstlisting}
def swap[A]: Either[A, A] => Either[A, A] = {
    case Left(a)    => Right(a)
    case Right(a)   => Left(a)
}
def merge[A]: Either[A, A] => A = {
    case Left(a)    => a
    case Right(a)   => a
}
\end{lstlisting}

\vspace{-1\baselineskip}
\end{wrapfigure}%

~\vspace{-1.2\baselineskip}
\[
\text{swap}\triangleq\begin{array}{|c||cc|}
 & A & A\\
\hline A & \bbnum 0 & \text{id}\\
A & \text{id} & \bbnum 0
\end{array}\quad,\quad\quad\text{merge}\triangleq\begin{array}{|c||c|}
 & A\\
\hline A & \text{id}\\
A & \text{id}
\end{array}\quad.
\]
\vspace{-0.4\baselineskip}

We can quickly prove by matrix multiplication that $\text{swap}\bef\text{swap}=\text{id}$
and $\text{swap}\bef\text{merge}=\text{merge}$:
\begin{align*}
 & \text{swap}\bef\text{swap}=\,\begin{array}{||cc|}
\bbnum 0 & \text{id}\\
\text{id} & \bbnum 0
\end{array}\,\bef\,\begin{array}{||cc|}
\bbnum 0 & \text{id}\\
\text{id} & \bbnum 0
\end{array}\,=\,\begin{array}{||cc|}
\text{id}\bef\text{id} & \bbnum 0\\
\bbnum 0 & \text{id}\bef\text{id}
\end{array}\,=\,\begin{array}{||cc|}
\text{id} & \bbnum 0\\
\bbnum 0 & \text{id}
\end{array}\,=\text{id}\quad,\\
 & \text{swap}\bef\text{merge}=\,\begin{array}{||cc|}
\bbnum 0 & \text{id}\\
\text{id} & \bbnum 0
\end{array}\,\bef\,\begin{array}{||c|}
\text{id}\\
\text{id}
\end{array}\,=\,\begin{array}{||c|}
\text{id}\bef\text{id}\\
\text{id}\bef\text{id}
\end{array}\,=\,\begin{array}{||c|}
\text{id}\\
\text{id}
\end{array}\,=\text{merge}\quad.
\end{align*}
Note that the ``identity matrix'' is the identity function for any
disjunctive type, e.g.~$A+B+C$:
\[
\begin{array}{|c||ccc|}
 & A & B & C\\
\hline A & \text{id} & \bbnum 0 & \bbnum 0\\
B & \bbnum 0 & \text{id} & \bbnum 0\\
C & \bbnum 0 & \bbnum 0 & \text{id}
\end{array}\,=\text{id}^{:A+B+C\rightarrow A+B+C}\quad.
\]

The type constructor $E^{A}\triangleq A+A$ is a functor whose lifting
is defined by

\begin{wrapfigure}{L}{0.6\columnwidth}%
\vspace{-0.8\baselineskip}
\begin{lstlisting}
def fmap[A, B](f: A => B): Either[A, A] => Either[B, B] = {
    case Left(a)    => Left(f(a))
    case Right(a)   => Right(f(a))
}
\end{lstlisting}

\vspace{-1.65\baselineskip}
\end{wrapfigure}%

~\vspace{-1.45\baselineskip}
\[
(f^{:A\rightarrow B})^{\uparrow E}\triangleq\begin{array}{|c||cc|}
 & B & B\\
\hline A & f & \bbnum 0\\
A & \bbnum 0 & f
\end{array}\quad.
\]
\vspace{-0.7\baselineskip}

Now we can formulate a law for the \lstinline!merge! function, called
the ``naturality law'':
\[
(f^{:A\rightarrow B})^{\uparrow E}\bef\text{merge}^{:B+B\rightarrow B}=\text{merge}^{:A+A\rightarrow A}\bef f^{:A\rightarrow B}\quad.
\]
Proving this law is a simple matrix calculation:
\begin{align*}
{\color{greenunder}\text{left-hand side}:}\quad & f^{\uparrow E}\bef\text{merge}=\,\begin{array}{||cc|}
f & \bbnum 0\\
\bbnum 0 & f
\end{array}\,\bef\,\begin{array}{||c|}
\text{id}\\
\text{id}
\end{array}\,=\,\begin{array}{||c|}
f\bef\text{id}\\
f\bef\text{id}
\end{array}\,=\,\begin{array}{||c|}
f\\
f
\end{array}\quad,\\
{\color{greenunder}\text{right-hand side}:}\quad & \text{merge}\bef f=\,\begin{array}{||c|}
\text{id}\\
\text{id}
\end{array}\,\bef\gunderline f=\,\begin{array}{||c|}
\text{id}\\
\text{id}
\end{array}\,\bef\,\begin{array}{||c|}
f\end{array}\,=\,\begin{array}{||c|}
\text{id}\bef f\\
\text{id}\bef f
\end{array}\,=\,\begin{array}{||c|}
f\\
f
\end{array}\quad.
\end{align*}
In the last line we replaced $f$ by a $1\times1$ matrix, $\,\begin{array}{||c|}
f\end{array}\;$, in order to apply matrix multiplication.

Matrix rows and columns can be split or merged when necessary to accommodate
various disjunctive types. As an example, let us verify the ``associativity
law'' of \lstinline!merge!,
\[
(\text{merge}^{:A+A\rightarrow A})^{\uparrow E}\bef\text{merge}^{:A+A\rightarrow A}=\text{merge}^{:(A+A)+(A+A)\rightarrow A+A}\bef\text{merge}^{:A+A\rightarrow A}\quad.
\]
Both sides of this law are functions of type $A+A+A+A\rightarrow A$.
To transform the left-hand side, we use the definition of $^{\uparrow E}$
and write
\[
\text{merge}^{\uparrow E}\bef\text{merge}=\,\begin{array}{|c||cc|}
 & A & A\\
\hline A+A & \text{merge} & \bbnum 0\\
A+A & \bbnum 0 & \text{merge}
\end{array}\,\bef\,\begin{array}{|c||c|}
 & A\\
\hline A & \text{id}\\
A & \text{id}
\end{array}\,=\,\begin{array}{|c||c|}
 & A\\
\hline A+A & \text{merge}\\
A+A & \text{merge}
\end{array}\quad.
\]
However, we have not substituted the definition of \lstinline!merge!
into the matrix. To do that, we need to expand the rows of the matrix
to accommodate the full disjunctive type $A+A+A+A$:
\[
\text{merge}^{\uparrow E}\bef\text{merge}=\,\begin{array}{|c||c|}
 & A\\
\hline A+A & \text{merge}\\
A+A & \text{merge}
\end{array}\,=\,\begin{array}{|c||c|}
 & A\\
\hline A & \text{id}\\
A & \text{id}\\
A & \text{id}\\
A & \text{id}
\end{array}\quad.
\]
Now we compute the right-hand side of the law by substituting the
code of \lstinline!merge!:
\[
\text{merge}\bef\text{merge}=\,\begin{array}{|c||c|}
 & A+A\\
\hline A+A & \text{id}\\
A+A & \text{id}
\end{array}\,\bef\,\begin{array}{|c||c|}
 & A\\
\hline A & \text{id}\\
A & \text{id}
\end{array}\quad.
\]
We cannot proceed with matrix multiplication because the dimensions
of the matrices do not match. We need to expand the rows and the columns
of the first matrix; then we compute
\[
\begin{array}{|c||c|}
 & A+A\\
\hline A+A & \text{id}\\
A+A & \text{id}
\end{array}\,\bef\,\begin{array}{|c||c|}
 & A\\
\hline A & \text{id}\\
A & \text{id}
\end{array}\,=\begin{array}{|c||cc|}
 & A & A\\
\hline A & \text{id} & \bbnum 0\\
A & \bbnum 0 & \text{id}\\
A & \text{id} & \bbnum 0\\
A & \bbnum 0 & \text{id}
\end{array}\,\bef\,\begin{array}{|c||c|}
 & A\\
\hline A & \text{id}\\
A & \text{id}
\end{array}\,=\,\begin{array}{|c||c|}
 & A\\
\hline A & \text{id}\\
A & \text{id}\\
A & \text{id}\\
A & \text{id}
\end{array}\quad.
\]
This proves the law (and also helps visualize how the transformations
work with various types).

In some cases, we cannot fully split the rows or the columns of a
matrix. For instance, if we are calculating with an arbitrary function
$f^{:\bbnum 1+A\rightarrow\bbnum 1+B}$, we cannot write this function
in a form of a $2\times2$ matrix because we do not know which parts
of the disjunction this function will return (the code of the function
$f$ is arbitrary and unknown). At most, we could split the \emph{rows}
by writing the function $f$ as a product of unknown functions $g^{:\bbnum 1\rightarrow\bbnum 1+B}$
and $h^{:A\rightarrow\bbnum 1+B}$:
\[
f=\,\begin{array}{|c||c|}
 & \bbnum 1+B\\
\hline \bbnum 1 & g\\
A & h
\end{array}\quad.
\]
The single column of this matrix remains unsplit. Either that column
will remain unsplit throughout the derivation, or additional information
about $f$, $g$, or $h$ will allow us to split the column.

Finally, there are two tricks that do not follow from the matrix intuition
and may sometimes simplify a disjunctive function.\footnote{These tricks are adapted from Section~2.8 of the book ``Program
design by calculation'' (draft from October 2019), see \texttt{\href{http://www4.di.uminho.pt/~jno/ps/pdbc.pdf}{http://www4.di.uminho.pt/$\sim$jno/ps/pdbc.pdf}}}

\paragraph{Ignored arguments}

If all rows of the disjunctive function ignore their arguments and
always return the same results, we may collapse all rows into one,
as shown in this example:

\begin{wrapfigure}{L}{0.5\columnwidth}%
\vspace{-0.2\baselineskip}
\begin{lstlisting}
def same[A]: Either[A, Option[A]] => Option[A] = {
    case Left(a)          => None
    case Right(None)      => None
    case Right(Some(a))   => None
}
\end{lstlisting}
\vspace{-3\baselineskip}
\end{wrapfigure}%

~\vspace{-1.4\baselineskip}
\begin{align*}
 & \text{same}^{:A+\bbnum 1+A\rightarrow\bbnum 1+A}=\,\begin{array}{|c||cc|}
 & \bbnum 1 & A\\
\hline A & \_\rightarrow1 & \bbnum 0\\
\bbnum 1 & \_\rightarrow1 & \bbnum 0\\
A & \_\rightarrow1 & \bbnum 0
\end{array}\\
 & =\,\begin{array}{|c||cc|}
 & \bbnum 1 & A\\
\hline A+\bbnum 1+A & \_\rightarrow1 & \bbnum 0
\end{array}\quad.
\end{align*}
\vspace{-1\baselineskip}

A more general formula for arbitrary functions $f^{:X\rightarrow C}$
is
\[
x^{:X}\rightarrow p^{:A+B}\triangleright\,\begin{array}{|c||c|}
 & C\\
\hline A & \_\rightarrow f(x)\\
B & \_\rightarrow f(x)
\end{array}\,=x^{:X}\rightarrow f(x)=f\quad.
\]
In this case, we can completely collapse the matrix to an ordinary
(non-disjunctive) function.

\paragraph{Simplification of diagonal pair products}

Consider the pair product of two disjunctive functions such as $f^{:A+B\rightarrow R}$
and $g^{:P+Q\rightarrow S}$. Computing $f\boxtimes g$ in the matrix
notation requires, in general, to split the rows and the columns of
the matrices because the type of $f\boxtimes g$ is 
\begin{align*}
f\boxtimes g & :(A+B)\times(P+Q)\rightarrow R\times S\\
 & \cong A\times P+A\times Q+B\times P+B\times Q\rightarrow R\times S\quad.
\end{align*}
So, the pair product of two $2\times1$ matrices is written \emph{in
general} as a $4\times1$ matrix:
\[
\text{for any }f\triangleq\begin{array}{|c||c|}
 & R\\
\hline A & f_{1}\\
B & f_{2}
\end{array}\quad\text{and}\quad g\triangleq\begin{array}{|c||c|}
 & S\\
\hline P & g_{1}\\
Q & g_{2}
\end{array}\quad,\quad\text{we have }\quad f\boxtimes g=\,\begin{array}{|c||c|}
 & R\times S\\
\hline A\times P & f_{1}\boxtimes g_{1}\\
A\times Q & f_{1}\boxtimes g_{2}\\
B\times P & f_{2}\boxtimes g_{1}\\
B\times Q & f_{2}\boxtimes g_{2}
\end{array}\quad.
\]

A simplification trick exists when the pair product is composed with
the diagonal function $\Delta$:
\[
\Delta\bef(f\boxtimes g)=\Delta^{:A+B\rightarrow(A+B)\times(A+B)}\bef(f^{:A+B\rightarrow R}\boxtimes g^{:A+B\rightarrow S})=p\rightarrow f(p)\times g(p)\quad.
\]
This ``diagonal pair product'' is well-typed only if $f$ and $g$
have the same argument types (so, $A=P$ and $B=Q$). It turns out
that the function $\Delta\bef(f\boxtimes g)$ can be written as a
$2\times1$ matrix, i.e.~we do not need to split the rows:
\[
\text{for any }f\triangleq\begin{array}{|c||c|}
 & R\\
\hline A & f_{1}\\
B & f_{2}
\end{array}\quad\text{and}\quad g\triangleq\begin{array}{|c||c|}
 & S\\
\hline A & g_{1}\\
B & g_{2}
\end{array}\quad,\quad\text{we have }\quad\Delta\bef(f\boxtimes g)=\,\begin{array}{|c||c|}
 & R\times S\\
\hline A & \Delta\bef(f_{1}\boxtimes g_{1})\\
B & \Delta\bef(f_{2}\boxtimes g_{2})
\end{array}\quad.
\]
The rules of matrix multiplication do not help in deriving this formula.
We resort to a more basic approach: show that both sides are equal
when applied to arbitrary values $p$ of type $A+B$:
\[
p^{:A+B}\triangleright\Delta\bef(f\boxtimes g)=f(p)\times g(p)\overset{?}{=}p\triangleright\,\begin{array}{|c||c|}
 & R\times S\\
\hline A & f_{1}\boxtimes g_{1}\\
B & f_{2}\boxtimes g_{2}
\end{array}\quad.
\]
Applying the left-hand side to $p\triangleq a^{:A}+\bbnum 0^{:B}$,
we get
\begin{align*}
 & f(p)\times g(p)=\big((a^{:A}+\bbnum 0^{:B})\triangleright f\big)\times\big((a^{:A}+\bbnum 0^{:B})\triangleright g\big)\\
 & \quad=\big(\,\begin{array}{|cc|}
a & \bbnum 0\end{array}\,\triangleright\,\begin{array}{||c|}
f_{1}\\
f_{2}
\end{array}\,\big)\times\big(\,\begin{array}{|cc|}
a & \bbnum 0\end{array}\,\triangleright\,\begin{array}{||c|}
g_{1}\\
g_{2}
\end{array}\,\big)=\big(a\triangleright f_{1}\big)\times\big(a\triangleright g_{1}\big)=f_{1}(a)\times g_{1}(a)\quad.
\end{align*}
Applying the right-hand side to the same $p$, we get
\begin{align*}
 & \gunderline p\triangleright\,\begin{array}{||c|}
\Delta\bef(f_{1}\boxtimes g_{1})\\
\Delta\bef(f_{2}\boxtimes g_{2})
\end{array}\,=\,\begin{array}{|cc|}
a & \bbnum 0\end{array}\,\triangleright\,\begin{array}{||c|}
\Delta\bef(f_{1}\boxtimes g_{1})\\
\Delta\bef(f_{2}\boxtimes g_{2})
\end{array}=\gunderline{a\triangleright\Delta}\bef(f_{1}\boxtimes g_{1})\\
 & \quad=(a\times a)\triangleright(f_{1}\boxtimes g_{1})=f_{1}(a)\times g_{1}(a)\quad.
\end{align*}
A similar calculation with $p\triangleq\bbnum 0^{:A}+b^{:B}$ shows
that both sides are equal to $f_{2}(b)\times g_{2}(b)$.

\subsection{Derivations involving unknown functions with laws}

A more challenging task is to derive an equation that uses arbitrary
functions about which we only know that they satisfy certain given
laws. Such derivations usually proceed by trying to transform the
code until the given laws can be applied.

As an example, let us derive the property that $L^{A}\triangleq A\times F^{A}$
is a functor if $F^{\bullet}$ is known to be a functor. We are in
the situation where we only know that the function $\text{fmap}_{F}$
exists and satisfies the functor law, but we do not know the code
of $\text{fmap}_{F}$. Let us discover the derivation step by step.

First, we need to define $\text{fmap}_{L}$. We use the lifting notation
$^{\uparrow F}$ and write, for any $f^{:A\rightarrow B}$,
\begin{lstlisting}
def fmap_L[A, B](f: A => B): ((A, F[A])) => (B, F[B]) = { case (a, p) => (f(a), p.map(f)) }
\end{lstlisting}
\[
f^{\uparrow L}\triangleq a^{:A}\times p^{:F^{A}}\rightarrow f(a)\times(p\triangleright f^{\uparrow F})\quad.
\]
To verify the functor identity law for $L$:
\begin{align*}
{\color{greenunder}\text{expect to equal }\text{id}:}\quad & \text{id}^{\uparrow L}=a^{:A}\times p^{:F^{A}}\rightarrow\text{id}\,(a)\times(p\triangleright\text{id}^{\uparrow F})\quad.
\end{align*}
At this point, the only things we can simplify are the identity functions
applied to arguments. We know that $F$ is a lawful functor; therefore,
$\text{id}^{\uparrow F}=\text{id}$. So we continue the derivation,
omitting types:
\begin{align*}
{\color{greenunder}\text{expect to equal }\text{id}:}\quad & \text{id}^{\uparrow L}=a\times p\rightarrow\gunderline{\text{id}\,(a)}\times(p\triangleright\gunderline{\text{id}^{\uparrow F}})\\
{\color{greenunder}\text{identity law of }F:}\quad & =a\times p\rightarrow a\times(\gunderline{p\triangleright\text{id}})\\
{\color{greenunder}\text{apply function}:}\quad & =a\times p\rightarrow a\times p=\text{id}\quad.
\end{align*}

To verify the functor composition law for $L$, we assume two arbitrary
functions $f^{:A\rightarrow B}$ and $g^{:B\rightarrow C}$:
\begin{align*}
{\color{greenunder}\text{expect to equal }(f\bef g)^{\uparrow L}:}\quad & f^{\uparrow L}\bef g^{\uparrow L}=\big(a\times p\rightarrow f(a)\times f^{\uparrow F}(p)\big)\bef\big(b\times q\rightarrow g(b)\times g^{\uparrow F}(q)\big)\quad.
\end{align*}
At this point, we pause and try to see how we might proceed. We do
not know anything about $f$ and $g$, so we cannot evaluate $f(a)$
or $f^{\uparrow F}(p)$. We also do not have the code of $^{\uparrow F}$
(i.e.~of $\text{fmap}_{F}$). The only information we have about
these functions is that $F$'s composition law holds,
\begin{equation}
f^{\uparrow F}\bef g^{\uparrow F}=(f\bef g)^{\uparrow F}\quad.\label{eq:composition-law-F-derivation1}
\end{equation}
We could use this law only if we somehow bring $f^{\uparrow F}$ and
$g^{\uparrow F}$ together in the formula. The only way forward is
to compute the function composition of the two functions whose code
we \emph{do} have:
\begin{align*}
 & \big(a\times p\rightarrow f(a)\times f^{\uparrow F}(p)\big)\bef\big(b\times q\rightarrow g(b)\times g^{\uparrow F}(q)\big)\\
 & =a\times p\rightarrow g(f(a))\times g^{\uparrow F}(f^{\uparrow F}(p))\quad.
\end{align*}
In order to use the law~(\ref{eq:composition-law-F-derivation1}),
we need to rewrite this code via the composition $f\bef g$. We notice
that we indeed find exactly those function compositions:
\[
g(f(a))\times g^{\uparrow F}(f^{\uparrow F}(p))=(a\triangleright f\bef g)\times(p\triangleright f^{\uparrow F}\bef g^{\uparrow F})\quad.
\]
We can now apply the composition law of $F$ and write up the complete
derivation, adding hints:
\begin{align*}
{\color{greenunder}\text{expect to equal }(f\bef g)^{\uparrow L}:}\quad & f^{\uparrow L}\bef g^{\uparrow L}=\big(a\times p\rightarrow f(a)\times f^{\uparrow F}(p)\big)\bef\big(b\times q\rightarrow g(b)\times g^{\uparrow F}(q)\big)\\
{\color{greenunder}\text{compute composition}:}\quad & =a\times p\rightarrow\gunderline{g(f(a))}\times\gunderline{g^{\uparrow F}(f^{\uparrow F}(p))}\\
{\color{greenunder}\triangleright\text{-notation}:}\quad & =a\times p\rightarrow(a\triangleright f\bef g)\times\big(p\triangleright\gunderline{f^{\uparrow F}\bef g^{\uparrow F}}\big)\\
{\color{greenunder}\text{composition law of }F:}\quad & =a\times p\rightarrow(a\triangleright\gunderline{f\bef g})\times\big(p\triangleright(\gunderline{f\bef g})^{\uparrow F}\big)\\
{\color{greenunder}\text{definition of }^{\uparrow L}:}\quad & =(f\bef g)^{\uparrow L}\quad.
\end{align*}

The derivation becomes significantly shorter if we use the pair product
to define $^{\uparrow L}$:
\[
f^{\uparrow L}\triangleq\text{id}\boxtimes f^{\uparrow F}\quad.
\]
For instance, verifying the identity law looks like this:
\[
\text{id}^{\uparrow L}=\text{id}\boxtimes\text{id}^{\uparrow F}=\text{id}\boxtimes\text{id}=\text{id}\quad.
\]
This technique was used in the proof of Statement~\ref{subsec:functor-Statement-functor-product}.
The cost of having a shorter proof is the need to remember the properties
of the $\boxtimes$ operation, which is not often used in derivations.

\subsection{Exercises\index{exercises}}

\subsubsection{Exercise \label{subsec:Exercise-reasoning-1-4}\ref{subsec:Exercise-reasoning-1-4}}

Show using matrix calculations that $\text{swap}\bef\text{swap}=\text{id}$,
where \lstinline!swap! is the function defined in Section~\ref{subsec:Working-with-disjunctive-functions}.

\subsubsection{Exercise \label{subsec:Exercise-reasoning-1-6}\ref{subsec:Exercise-reasoning-1-6}}

Now consider a different function \lstinline!swap! with the type
signature

\begin{wrapfigure}{L}{0.63\columnwidth}%
\vspace{-0.65\baselineskip}
\begin{lstlisting}
def swap[A, B]: ((A, B)) => (B, A) = { case (a, b) => (b, a) }
\end{lstlisting}

\vspace{-0.25\baselineskip}
\end{wrapfigure}%

~\vspace{-1.25\baselineskip}
\[
\text{swap}^{A,B}\triangleq a^{:A}\times b^{:B}\rightarrow b\times a\quad.
\]
\vspace{-0.15\baselineskip}
Show that $\Delta\bef\text{swap}=\Delta$. Write out all types in
this law and draw a type diagram.

\subsubsection{Exercise \label{subsec:Exercise-reasoning-1-1}\ref{subsec:Exercise-reasoning-1-1}}

Given an arbitrary functor $F$, define the functor $L^{A}\triangleq F^{A}\times F^{A}$
and prove, for an arbitrary function $f^{:A\rightarrow B}$, the ``lifted
naturality'' law
\[
f^{\uparrow F}\bef\Delta=\Delta\bef f^{\uparrow L}\quad.
\]
Write out all types in this law and draw a type diagram.

\subsubsection{Exercise \label{subsec:Exercise-reasoning-1-5}\ref{subsec:Exercise-reasoning-1-5}}

Show that the types $(\bbnum 1+\bbnum 1)\times A$ and $A+A$ are
equivalent. One direction of this equivalence is given by a function
\lstinline!two[A]! with the type signature

\begin{wrapfigure}{L}{0.63\columnwidth}%
\vspace{-0.65\baselineskip}
\begin{lstlisting}
def two[A]: ((Either[Unit, Unit], A)) => Either[A, A] = ???
\end{lstlisting}

\vspace{-0.25\baselineskip}
\end{wrapfigure}%

~\vspace{-1.15\baselineskip}
\[
\text{two}^{A}:(\bbnum 1+\bbnum 1)\times A\rightarrow A+A\quad.
\]
\vspace{-0.35\baselineskip}
Implement that function and prove that it satisfies the ``naturality
law'': for any $f^{:A\rightarrow B}$,
\[
(\text{id}\boxtimes f)\bef\text{two}=\text{two}\bef f^{\uparrow E}\quad,
\]
where $E^{A}\triangleq A+A$ is the functor whose lifting $^{\uparrow E}$
was defined in Section~\ref{subsec:Working-with-disjunctive-functions}.
Write out the types in this law and draw a type diagram. 

\subsubsection{Exercise \label{subsec:Exercise-reasoning-1}\ref{subsec:Exercise-reasoning-1}}

Prove that the following laws hold for arbitrary $f^{:A\rightarrow B}$
and $g^{:C\rightarrow D}$:
\begin{align*}
{\color{greenunder}\text{left projection law}:}\quad & (f\boxtimes g)\bef\nabla_{1}=\nabla_{1}\bef f\quad,\\
{\color{greenunder}\text{right projection law}:}\quad & (f\boxtimes g)\bef\nabla_{2}=\nabla_{2}\bef g\quad.
\end{align*}


\subsubsection{Exercise \label{subsec:Exercise-reasoning-1-2}\ref{subsec:Exercise-reasoning-1-2}}

Given arbitrary functors $F$ and $G$, define the functor $L^{A}\triangleq F^{A}\times G^{A}$
and prove that for arbitrary $f^{:A\rightarrow B}$,
\[
f^{\uparrow L}\bef\nabla_{1}=\nabla_{1}\bef f^{\uparrow F}\quad.
\]
Write out the types in this law and draw a type diagram. 

\subsubsection{Exercise \label{subsec:Exercise-reasoning-1-3}\ref{subsec:Exercise-reasoning-1-3}}

Consider the functor $L^{A}$ defined as 
\[
L^{A}\triangleq\text{Int}\times\text{Int}+A\quad.
\]
Implement the functions \lstinline!fmap! and \lstinline!flatten!
(denoted $\text{ftn}_{L}$) and write their code in matrix notation:
\begin{align*}
(f^{:A\rightarrow B})^{\uparrow L} & :\text{Int}\times\text{Int}+A\rightarrow\text{Int}\times\text{Int}+B\quad,\\
\text{ftn}_{L} & :\text{Int}\times\text{Int}+\text{Int}\times\text{Int}+A\rightarrow\text{Int}\times\text{Int}+A\quad.
\end{align*}


\subsubsection{Exercise \label{subsec:Exercise-reasoning-1-3-1}\ref{subsec:Exercise-reasoning-1-3-1}{*}}

Show that \lstinline!flatten! (denoted $\text{ftn}_{L}$) from Exercise~\ref{subsec:Exercise-reasoning-1-3}
satisfies the naturality law: for any $f^{:A\rightarrow B}$,
\[
f^{\uparrow L\uparrow L}\bef\text{ftn}_{L}=\text{ftn}_{L}\bef f^{\uparrow L}\quad.
\]

