
\chapter{Sample problems}
\begin{enumerate}
\item Compute the smallest integer expressible as a sum of two cubed integers
in more than one way.
\item Read a text file, split it by spaces into words, and print the word
counts, sorted by decreasing count.
\item FPIS exercise 2.2: Check whether a sequence \lstinline!Seq[A]! is
sorted according to a given ordering function of type \lstinline!(A, A) => Boolean!.
\item FPIS exercise 3.24: Implement a function \lstinline!hasSubsequence!
that checks whether a \lstinline!List! contains another \lstinline!List!
as a subsequence. For instance, \lstinline!List(1,2,3,4)! would have
\lstinline!List(1,2)!, \lstinline!List(2,3)!, and \lstinline!List(4)!
as subsequences, among others. (Dynamic programming?)
\item (Bird, de Moor page 20) Derive the following identity between functions
$F^{A}\rightarrow F^{A}$, for any filterable functor $F$ and any
predicate $p^{:A\rightarrow\bbnum 2}$: 
\[
\text{filt}_{F}(p)=(\Delta\bef\text{id}\boxtimes p)^{\uparrow F}\bef\text{filt}_{F}(\pi_{2})\bef\pi_{1}^{\uparrow F}\quad.
\]
\item Define a monoid of partial functions with fixed types $P\rightarrow Q$.
\begin{lstlisting}
final case class PFM[P, Q](pf: PartialFunction[P, Q])
// After defining a monoid instance, the following code must work:
val p1 = PFM[Option[Int], String] { case Some(3) => "three" }
val p2 = PFM[Option[Int], String] {
  case Some(20)   => "twenty"
  case None       => "empty"
}
p1 |+| p2 // Must be the same as the concatenation of all `case` clauses.
\end{lstlisting}
\item Consider a typeclass called \textsf{``}\lstinline!Splittable!\textsf{''} for functors
$F^{\bullet}$ that have an additional method 
\[
\text{split}^{A,B}:F^{A+B}\rightarrow F^{A}+B
\]
with the non-degeneracy law for functions $F^{A}\rightarrow F^{A}$,
\[
(x^{:A}\rightarrow x+\bbnum 0^{:B})^{\uparrow F}\bef\text{split}=y^{:F^{A}}\rightarrow y+\bbnum 0^{:B}
\]
and the special associativity law for functions $F^{A+B+C}\rightarrow F^{A}+B+C$,
\[
\text{split}^{A+B,C}\bef\begin{array}{|c||cc|}
 & F^{A}+B & C\\
\hline F^{A+B} & \text{split}^{A,B} & \bbnum 0\\
C & \bbnum 0 & \text{id}
\end{array}=\text{split}^{A,B+C}\quad.
\]
Show that all polynomial functors $F^{\bullet}$ belong to this typeclass.
Show that exponential functors such as $F^{A}\triangleq Z\rightarrow A$
do not. 
\end{enumerate}

